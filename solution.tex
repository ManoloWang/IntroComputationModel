\documentclass[a4paper,twoside]{article}
\usepackage[margin=1.25in]{geometry}
\usepackage[UTF8,fontset=none]{ctex}
\usepackage{amssymb}
\usepackage{amsthm}
\usepackage{titlesec}
\usepackage{fancyhdr}
\usepackage{listings}
\usepackage{indentfirst}
\usepackage{caption}
\usepackage{tabularx}
\usepackage{float}
\usepackage{longtable}
%\setlength{\parindent}{2em}
%%%%%%%%%%%%%%%%%%%%%%% TeX defines %%%%%%%%%%%%%%%%%%%%%%%
% Listing Style
\lstset {
        basicstyle = \sf,
        language = Java,
        tabsize = 4,
        columns = flexible
}

% Enumeration separations
\usepackage{enumitem}
\usepackage{amsmath}
\usepackage{amssymb}
\usepackage{mathabx}
\setenumerate[1]{itemsep=3pt,partopsep=0pt,parsep=\parskip,topsep=5pt}
\setitemize[1]{itemsep=3pt,partopsep=0pt,parsep=\parskip,topsep=5pt}
\setdescription{itemsep=3pt,partopsep=0pt,parsep=\parskip,topsep=5pt}
\usepackage{sectsty}



%%%%%%%%%%%%%%%%%%%%%%% TOC defines %%%%%%%%%%%%%%%%%%%%%%%
% General terms
\newcommand{\problem}{\subsection}
\newenvironment{solution}{\begin{proof}[解]}{\end{proof}}

% Recursive functions
\newcommand{\IF}{\mathcal{IF}}
\newcommand{\BF}{\mathcal{BF}}
\newcommand{\EF}{\mathcal{EF}}
\newcommand{\PRF}{\mathcal{PRF}}
\newcommand{\RF}{\mathcal{RF}}
\newcommand{\GRF}{\mathcal{GRF}}
\newcommand{\pg}{\mathrm{pg}}
\newcommand{\rs}{\mathrm{rs}}
\newcommand{\eq}{\mathrm{eq}}
\newcommand{\ep}{\mathrm{ep}}

% Abacus machines
\newcommand{\AM}{\mathrm{AM}}
\renewcommand{\L}{\left\langle}
\newcommand{\R}{\right\rangle}
\newcommand{\Z}{\mathbf{Z}}
\newcommand{\Sub}{\mathrm{S}}
\newcommand{\A}{\mathrm{A}}
\newcommand{\cp}{\mathrm{copy}}
\newcommand{\mv}{\mathrm{move}}

% Lambda calculus
\newcommand{\bnf}{\beta\textrm{-nf}}
\newcommand{\subterm}{\textrm{Sub}}
\newcommand{\onestepbeta}{\rightarrow_{\beta}}
\newcommand{\manystepbeta}{\twoheadrightarrow_{\beta}}
\newcommand{\equivbeta}{=_{\beta}}
\newcommand{\Y}{\mathbf{Y}}
\newcommand{\K}{\mathbf{K}}
\newcommand{\D}{\mathbf{D}}
\newcommand{\Lambdaclosed}{\Lambda^\circ}
\newcommand{\churchnumber}[1]{\ulcorner{#1}\urcorner}
\newcommand{\NF}{\mathrm{NF}}

% Turing Machines
\newcommand{\machine}[1]{\boxed{\mathrm{#1}}}
\newcommand{\concat}{\mapstochar\Rightarrow}
\setCJKmainfont[Path=fonts/, BoldFont=simhei.ttf, ItalicFont=simkai.ttf]{simsun.ttf}
\makeatletter
\def\thickhline{%
  \noalign{\ifnum0=`}\fi\hrule \@height \thickarrayrulewidth \futurelet
   \reserved@a\@thickhline}
\def\@thickhline{\ifx\reserved@a\thickhline
               \vskip\doublerulesep
               \vskip-\thickarrayrulewidth
             \fi
      \ifnum0=`{\fi}}
\makeatother

\newlength{\thickarrayrulewidth}
\setlength{\thickarrayrulewidth}{2\arrayrulewidth}
\newcolumntype{Y}{>{\centering\arraybackslash}X}
% Page head
\pagestyle{fancy}
\renewcommand{\sectionmark}[1]{\markright{#1}}
\fancyhead{} % clear all fields
\fancyhead[CO]{《计算模型导引》作业}
\fancyhead[CE]{第{\thesection}章 \leftmark}
\fancyhead[LE,RO]{\thepage}
\fancyfoot{}

% Title
\title{\bf 《计算模型导引》作业}
\author{}
\date{}
\begin{document}

\begin{titlepage}
\maketitle
\thispagestyle{empty}
\end{titlepage}
\begin{center}
\bf\LARGE 声明
\end{center}

本习题解答与南京大学宋方敏教授编著的《计算模型导引》配套。

所有习题解答都是经过本人的思考,与周围人的讨论,以及来自大佬的讲解得到,所以整个题解并未完成所有的题目,也不能保证答案的严谨性和正确性,因此仅供参考。由于使用此答案造成的任何后果,请自行承担,本人概不负责。

本解答中许多灵感启发来自于蒋炎岩前辈提供的一份英文版参考解答,张强同学、李浩同学的解答,以及来自助教的解答,在此表示特别感谢。

同时殷切盼望各位大佬能对这份解答做出哪怕一点微小的贡献,包括但不限于:尚未解出题目的解答,不同于已有解答的另外的解题方法,对已有解答描述不够严谨甚至是错误的部分的补充、纠正或指出,对已有解答推理或描述不够清晰部分的补充或指出,公式中的错误,错别字等。任何的补充指正和提问,都感激不尽。
\thispagestyle{empty}
\newpage
\thispagestyle{empty}
\setcounter{page}{1}
% Chapter 1
\section{递归函数}
\problem{证明: 对于固定的$k$, 一元数论函数$x+k\in\BF$.}
\begin{proof}
令$f_k(x)=x+k$.
\begin{enumerate}
\item 当$k=0$时, $f_0(x)=x=P_1^1(x)\in\IF$, 命题成立;
\item 当$k=1$时, $f_1(x)=x+1=S(x)\in\IF$, 命题成立;
\item 假设当$k=k_0(k_0\geqslant 1,k_0\in\mathbb{N})$时$f_{k_0}\in\BF$, 那么:

当$k=k_0+1$时,$f_{k_0+1}=x+k_0+1=S(x+k_0)=S\circ f_{k_0}(x)=\mathrm{Comp}^1_1[S,f_{k_0}]\in\BF$.
\end{enumerate}

由数学归纳法可知, 命题成立.
\end{proof}
\problem{证明: 对任意$k\in\mathbb{N}^+,f:\mathbb{N}^k\to\mathbb{N}$, 若$f\in\BF$, 则存在$h$, 使得$$f(\vec{x})<\|\vec{x}\|+h.$$其中$\|x\|\equiv\max\{x_i:1\leqslant i\leqslant k\}$.}
\begin{proof}
分情况讨论.
\begin{enumerate}
\item 当$f\in\IF$时, 由于$S(x)=x+1<\|x\|+2,Z(x)=0<\|x\|+2,P(\vec{x})\leqslant\|\vec{x}\|<\|\vec{x}\|+2$, 所以令$h=2$即可让题中式子对于$f\in\IF$均成立;
\item 当$f\in\BF-\IF$时, 假设$f$的构造长度$\ell$满足$0\leqslant l \leqslant k$时, 命题均成立, 那么对这样的函数$f$, 必然存在一个自然数$h_k$使得$f(\vec{x})<\|\vec{x}\|+h_k$恒成立. 对于$f\in\IF$, 其构造长度可视为0, 故令$h_0=2$.

当$f$的构造长度为$k+1$时, 设其构造过程为数论函数序列$f_0,f_1,\cdots,f_k,f$, 那么对于任一函数$f_i(0\leqslant i\leqslant k)$, 其构造长度不大于$i$, 自然也不大于$k$, 所以有
\begin{equation}\label{eq1.2.1}
\begin{split}
	f&=\mathrm{Comp}_n^m[f_{i_0},f_{i_1},\cdots,f_{i_n}]\\
	 &=f_{i_0}(f_{i_1}(\vec{x}),\cdots,f_{i_n}(\vec{x}))\\
	 &<\|f_{i_1}(\vec{x}),\cdots,f_{i_n}(\vec{x})\|+h_k
\end{split}
\end{equation}
其中$f_{i_0},f_{i_1},\cdots,f_{i_n}$从$f_0,f_1,\cdots,f_k$中选择(可重复), 所以式(\ref{eq1.2.1})最后出现的$f_{i_1}(\vec{x}),$ $\cdots,$ $f_{i_n}(\vec{x})$的所有值均小于$\|\vec{x}\|+h_k$, 可得$\|f_{i_1}(\vec{x}),\cdots,f_{i_n}(\vec{x})\|<\|\vec{x}\|+h_k$, 因此$f<\|\vec{x}\|+2h_k$, 也就是说, 当$f$的构造长度$l$满足$0\leqslant l\leqslant k+1$时, $f(\vec{x})<\|\vec{x}\|+2h_k$恒成立, 即$h_{k+1}=2h_k$是满足要求的.
\end{enumerate}

又因为$h_0=2$, 可得$h_k=2^{k+1}$对于构造长度不大于$k$的函数都能使不等式$f(\vec{x})<\|\vec{x}\|+h_k$恒成立. 而对一个特定的$f\in\BF$, 它肯定有一个构造长度$\ell$, 所以令$h=2^{\ell+1}$即可使得题中不等式成立.
\end{proof}
\problem{证明: 二元数论函数$x+y\notin\BF$.}
\begin{proof}
由1.2的结论可知, 若二元数论函数$f\in\BF$, 则存在自然数$h$使得$f(x,y)<\max\{x,y\}+h$恒成立.

假设$x+y\in\BF$, 那么存在自然数$h$使得$x+y<\max\{x,y\}+h$恒成立. 因为$x+y=\max\{x,y\}+\min\{x,y\}$, 可推出该自然数$h$使得$\min\{x,y\}<h$对于任意的自然数$x,y$都成立, 这显然是错误的, 任给定一个$h$, 只需让$x,y$都大于$h$就可使不等式不成立. 所以$x+y\notin\BF$.
\end{proof}
\problem{证明: 二元数论函数$x\dotdiv y\notin\BF$.}
\begin{proof}
假设$x\dotdiv y\in\BF$, 因为$\mathrm{pred}=\mathrm{Comp}_2^1[\dotdiv,P_1^1,S\circ Z]$, 所以$\mathrm{pred}\in\BF$. 所以, 要证明出题目的命题, 只要证明$\mathrm{pred}\notin\BF$即可.

显然$\mathrm{pred}\notin\IF$, 那么若$\mathrm{pred}\in\BF$, 则$\mathrm{pred}$拥有最短构造过程, 设其为$f_0,f_1,\cdots,f_n,$ $\mathrm{pred}$. 显然,$\mathrm{pred}$是通过$f_0,f_1,\cdots,f_n$中的某些函数通过复合运算得来.

$f_0,f_1,\cdots,f_n$中不可能出现投影函数$P$, 首先, 由于$\mathrm{pred}$是一元函数, 如果序列中投影函数, 它们的定义域为$\mathbb{N}^k$. 这个投影函数如若使用, 就必须接受$k$个自然数, 这$k$个自然数就只能表示成$g_1(x),g_2(x),\cdots,g_k(x)$的形式, $g_i(x)\in\BF,1\leqslant i\leqslant k$. 那么问题在于, 投影函数只会固定选用其中一个, $P_i^k$就只会选$g_i(x)$, 那么根本不用$P_i^k$的参与, 只用$g_i(x)$就能参与之后的构造, 所以对于一元函数的\textbf{最短}构造过程来说, 投影函数是无用的. 那么$f_0,f_1,\cdots,f_n$要么就是$S, Z$, 要么就是由$S, Z$复合而来.

这样一来, 每一步构造都是$\mathbb{N}\to\mathbb{N}$. 在这时, 函数的复合满足结合律. 因此, $\mathrm{pred}$可表示成$F_1\circ F_2\circ \cdots \circ F_k$的形式, $F_i\in{S,Z}$.

而$Z\circ F_i\circ \cdots \circ F_k=0$恒成立, 因此上式可表示成$S^a\circ Z^b$, 其中, $a\in\mathbb{N},b\in\{0,1\}$且$a,b$不同时为0. 这里$$F^a\equiv \underbrace{F\circ F\circ \cdots \circ F}_{a}$$

当$a>0$时, 令$x=0$, 结果肯定不为0; 当$a=0$时, 式子就是$Z$, 同样不对. 因此$\mathrm{pred}\notin\BF$, 从而$x\dotdiv y\notin\BF$.
\end{proof}
\problem{设$\pg(x,y)=2^x(2y+1)\dotdiv 1$, 证明: 存在初等函数$K(x)$和$L(x)$使得
$$\protect\begin{aligned}
    K(\pg(x,y))=x,&\\
    L(\pg(x,y))=y,&\\
    \pg(K(z),L(z))=z.
\protect\end{aligned}$$}
\begin{proof}
    当$x,y\in\mathbb{N}$时, 可知$2^x(2y+1)\geqslant 1$恒成立, 因此$\pg(x,y)=2^x(2y+1)-1$(就是把$\dotdiv$换成了普通的减号$-$).

    又因为$2\nmid (2y+1)$, 所以可使$K(z)=\ep_0(z+1),L(z)=\frac12\left(\frac{z+1}{2^{\ep_0}(z+1)}-1\right)$使得$K(\pg(x,y))=x,L(\pg(x,y))=y$.

    且$$\pg(K(z),L(z))=2^{\ep_0(z+1)}\frac{z+1}{2^{\ep_0(z+1)}}-1=z.$$
\end{proof}
\problem{设$f:\mathbb{N}\to\mathbb{N}$, 证明: $f$可以作为配对函数的左函数当且仅当对任何$i\in\mathbb{N}$,$$|\{x\in\mathbb{N}:f(x)=i\}|=\aleph_0.$$}
\begin{proof}
    记$\mathcal{X}_i=\{x\in\mathbb{N}:f(x)=i\}$.

    \textbf{先证明必要性.}

    显然, $\mathcal{X}_i\subseteq\mathbb{N}$, 因此只需要证明该集合内的元素个数无限即可. \footnote{根据可数选择公理(axiom of countable choice), $\aleph_0$即自然数集等无限可数集的基数是所有无限集的基数里面最小的.}

    假设$\mathcal{X}_i$是有限集.

    考虑与之对应的配对函数$\pg:\mathbb{N}^2\to\mathbb{N}$和右函数$g:\mathbb{N}\to\mathbb{N}$, 记$\mathcal{Y}=\{g(x)|x\in\mathcal{X}_i\}$, 那么显然$\mathcal{Y}$也是有限集, 集合$\mathbb{N}-\mathcal{Y}$非空.

    任取$j\in\mathbb{N}-\mathcal{Y}$, 若$\pg(i,j)\in\mathcal{X}_i$, 则$j=g(\pg(i,j))\in\mathcal{Y}$, 与$j\in\mathbb{N}-\mathcal{Y}$矛盾; 若$\pg(i,j)\notin\mathcal{X}_i$, 则$f(\pg(i,j)) \neq i$, 与配对函数的定义矛盾. 故$\mathcal{X}_i$是无限集, 因而命题成立.

    \textbf{再证明充分性.}

    对任意的$i\in\mathbb{N}$, 都有$|\{x\in\mathbb{N}:f(x)=i\}|=\aleph_0$, 那么$\mathcal{X}_i$可以与$\mathbb{N}$建立一个双射, 记为函数$F_i:\mathbb{N}\to\mathcal{X}_i$及其反函数$F_i^{-1}:\mathcal{X}_i\to\mathbb{N}$.

    现在, 可以定义$g:\mathbb{N}\to\mathbb{N}$如下:
    $$g(x)=\begin{cases}
        j,&\textrm{若} x=F_i(j),\\
        0,&\textrm{否则}.
    \end{cases}$$
	
    那么, 当$x=F_i(j)$时, 因为$F_i(j)\in\mathcal{X}_i$, $f(x)=i$成立; 同时$g(x)=j$成立. 所以, 令$\pg(i,j)=F_i(j)$即可, $f$即为$\pg$的左函数.
\end{proof}
\problem{证明: 从本原函数出发, 经复合和算子$\displaystyle\prod_{i=n}^m[\cdot]$可以生成所有的初等函数. 这里$$\prod_{i=n}^m [f(i)] = 
    \protect\begin{cases}
        f(n)\cdot f(n+1)\cdot\quad\cdots\quad\cdot f(m), & \textrm{若}m\geqslant n,\\
        1, & \textrm{若}m<n.
    \protect\end{cases} $$}
\begin{proof}
    由初等函数的定义和引理1.12, 只需要证明$\ddot{-}$, 有界迭加算子$\Sigma[\cdot]$和有界迭乘算子$\Pi[\cdot]$能用复合以及算子$\displaystyle\prod_{i=n}^m[\cdot]$表示.

    有界迭乘算子是算子$\displaystyle\prod_{i=n}^m[\cdot]$中$n=0$的特例.

    首先不难构造出以下几个函数:

    指数函数\footnote{这里约定$0^0=1$.}$$x^y=\prod_{i=1}^y[P_1^1(x)]$$
	
    取反函数$$N(x)=\prod_{i=1}^x[Z(i)]$$
	
    $x\leqslant y$的特征函数$$\mathrm{leq}(x,y)=\prod_{i=x}^y[Z(i)]$$
	
    $x\geqslant y$的特征函数$$\mathrm{geq}(x,y)=\prod_{i=y}^x[Z(i)]$$

    然后:

    $x=y$的特征函数$$\eq(x,y)=\mathrm{leq}(x,y)^{N(\mathrm{geq}(x,y))}$$
    $$2^x=\prod_{i=1}^x[S\circ S\circ Z(i)]$$
    
    令$$\log(x)=\begin{cases}
        \log_2x,&\textrm{若}\log_2x\in\mathbb{N},\\
        1,&\textrm{否则.}
    \end{cases}$$
	
    那么
    $$\log(x)=\prod_{i=0}^x[i^{N(\eq(2^i,x))}]$$
    $$\log(2^x)=\prod_{i=0}^{2^x}[i^{N(\eq(2^i,2^x))}]$$

    所以
    $$\sum_{i=n}^m f(i,\vec{x})=\log(2^{\sum_{i=n}^m f(i,\vec{x})})=\log(\prod_{i=n}^m 2^{f(i,\vec{x})})$$

    $$x\ddot{-}y=\sum_{i=y+1}^x[S\circ Z(i)] + \sum_{i=x+1}^y[S\circ Z(i)]$$

    综上, 证明完毕.
\end{proof}
\problem{设$$M(x)=\protect\begin{cases}
    M(M(x+11)), & \textrm{若}x\leqslant 100,\\
    x-10, & \textrm{若}x>100,
\protect\end{cases}$$证明:
$$M(x)=
\protect\begin{cases}
    91, & \textrm{若}x\leqslant 100,\\
    x-10, & \textrm{否则}.
\protect\end{cases}$$}
\begin{proof}
    显然, 我们只要证明当$0\leqslant x\leqslant 100$时, $M(x)=91$即可.

    当$90\leqslant x \leqslant 100$时, $M(x)=M(M(x+11))=M(x+1)$, 因此$M(90)=M(91)=\cdots=M(100)=M(101)=91$, 注意$M(91)=91$;

    当$0\leqslant x \leqslant 100$时, 存在自然数$k$使得$90\leqslant x+11k\leqslant 100$成立. 因此$M(x)=M^2(x+11\cdot1)=M^{k+1}(x+11k)=M^kM(x+11k)=M^k(91)=91$.
\end{proof}
\problem{证明:
$$\min x\leqslant n. [f(x,\vec{y})]=n\dotdiv\max x\leqslant n. [f(n\dotdiv x,\vec{y})],$$
$$\max x\leqslant n. [f(x,\vec{y})]=n\dotdiv\min x\leqslant n. [f(n\dotdiv x,\vec{y})].$$}
\begin{proof}
    分情况讨论.
    \begin{enumerate}
        \item 当$f(x,\vec{y})$在$[0,n]$有零点时, 设$\min x\leqslant n. [f(x,\vec{y})]=k$.

        那么有$f(k)=0$, 从而$f(n\dotdiv(n\dotdiv k))=0$, $n\dotdiv k$是函数$f(n\dotdiv x)$的零点.
    
        因为$k$为$f(k)$最小的零点, 那么$n\dotdiv k$是$f(n\dotdiv x)$在$[0,n]$的最大零点. 因为如果有比$n\dotdiv k$大的$f(n\dotdiv x)$的零点$z$存在, 那么$f(n\dotdiv z)=0$且$n\dotdiv z<k$, 与$k$为$f(x)$最小零点矛盾.
    
        于是$\min x\leqslant n. [f(x,\vec{y})]=n\dotdiv\max x\leqslant n. [f(n\dotdiv x,\vec{y})]$成立.

        \item 当$f(x,\vec{y})$在$[0,n]$没有零点时, $\min x\leqslant n. [f(x,\vec{y})]=n$, $\max x\leqslant n. [f(n\dotdiv x,\vec{y})] = 0$, 因此式子依然成立.
    \end{enumerate}
	
    同理可证, $\max x\leqslant n. [f(x,\vec{y})]=n\dotdiv\min x\leqslant n. [f(n\dotdiv x,\vec{y})]$成立.
\end{proof}
\problem{证明: $\EF$对有界max-算子封闭.}
\begin{proof}
    因为$$\max x\leqslant n. [f(x,\vec{y})]=n\dotdiv\min x\leqslant n. [f(n\dotdiv x,\vec{y})],$$
	
    又因为引理1.12(5)和引理1.14, $\EF$对$\dotdiv$和有界$\mu$-算子封闭, 故$\EF$对有界max-算子封闭.
\end{proof}
\problem{Euler函数$\varphi:\mathbb{N}\to\mathbb{N}$定义为 $$\varphi(n)=|\{x:0<x\leqslant n \land \gcd(x,n)=1\}|,$$ 即$\varphi(n)$表示小于等于$n$且与$n$互素的正整数个数. 例如$\varphi(1)=1$, 因为1与其本身互素; $\varphi(9)=6$, 因为9与1,2,4,5,7,8互素. 证明: $\varphi\in\EF$.}
\begin{proof}
    $$\varphi(n)=\sum_{i=1}^n N(\gcd(i,n)\dotdiv 1)$$
    $$\gcd(x,y)=\max\{z:(z|x)\land (z|y)\}=\max z\leqslant x. [\rs(x,z) + \rs(y,z)]$$
	
    综上, $\varphi\in\EF$.
\end{proof}
\problem{设$h(x)$为$x$的最大素因子的下标, 约定$h(0)=0,h(1)=0$. 例如$h(88)=4$, 因为$88=2^3\times 11$的最大素因子11是第4个素数$p_4$, 其下标为4. 证明: $h\in\EF$.}
\begin{proof}
    由命题1.30知, $\ep_n(x)\in\EF$.
    而$h(x)=\max y\leqslant x. [1\dotdiv\ep_y(x)]$, 所以$h\in\EF$.
\end{proof}
\problem{设$f:\mathbb{N}\to\mathbb{N}$满足:$$\protect\begin{aligned}
    f(0)=&1,\\
    f(1)=&1,\\
    f(x+2)=&f(x)+f(x+1),
\protect\end{aligned}$$证明:\\
    (1) $f\in\PRF$;\\
    (2) $f\in\EF$.}
\begin{proof}
    令$F(n)=2^{f(n)}3^{f(n+1)}$, 有$F(0)=2^1\cdot 3^1=6$,
    $$\begin{aligned}
        F(n+1)&=2^{f(n+1)}3^{f(n+2)}\\
        &=2^{f(n+1)}3^{f(n+1)+f(n)}\\
        &=2^{\ep_1(F(n))}3^{\ep_1(F(n))+\ep_0(F(n))}\\
        &=H(F(n))
    \end{aligned}$$
	
    这里$H(x)=2^{\ep_1(x)}3^{\ep_1(x)+\ep_0(x)}$,由于$x^y,+,\times,\ep\in\EF$, 所以$H(x)\in\EF$. 自然$H(x)\in\PRF$.

    又因为$F(0)=6,F(n+1)=H(F(n))=H\circ P_2^2(n,F(n))$, 所以$F=\mathrm{Prim}^0(6,H\circ P_2^2)\in\PRF$. 因为$f(n)=\ep_0(F(n))$, 所以$f\in\PRF$.

    现在证明$f\in\EF$.

    首先证明$f(x)\leqslant 2^x$.
    
    因为$f(0)=1\leqslant2^0,f(1)=1\leqslant2^1,f(x+2)=f(x)+f(x+1)\leqslant2^x+2^{x+1}\leq2^{x+1}+2^{x+1}=2^{x+2}$, 所以由数学归纳法可证明$f(x)\leqslant 2^x$.

    那么令$G(n)=2^{2^n}3^{2^{n+1}}$, 有$F(n)=2^{f(n)}3^{f(n+1)}\leqslant2^{2^n}3^{2^{n+1}}=G(n)$, 而$G(n)\in\EF$, 那么可以如下构造一番:

    $$F\circ P^2_2(x, 0)=F(0)=6=S^6Z(x)$$
    $$F\circ P^2_2(x, n+1)=H\circ P_3^3(x, n, F\circ P^2_2(x,n))$$
    $$G\circ P^2_2(x, n)\equiv G(n)$$

    显然$S^6Z\in\EF,H\circ P_3^3\in\EF,G\circ P^2_2\in\EF$(前面已经证明$H\in\EF$), 所以根据定理1.31, $F\circ P^2_2\in\EF$, 那么$F=P_1^1\circ (F\circ P^2_2)\in\EF$, 因为$f(n)=\ep_0(F(n))$, 所以$f\in\EF$.
\end{proof}
\problem{设数论谓词$Q(x,y,z,v)$定义为
$$Q(x,y,z,v)\equiv p(\langle x,y,z \rangle)|v,$$
其中$p(n)$代表第$n$个素数, $\langle x,y,z \rangle$是$x,y,z$的G\"odel编码. 证明: $Q(x,y,z,v)$是初等的.}
\begin{proof}
    $Q(x,y,z,v)$的特征函数为
    $$\chi_Q(x,y,z,v)=N^2(\rs(v,p(\langle x,y,z \rangle))),$$
	
    因为$\langle x,y,z \rangle=2^x3^y5^z\in\EF,N^2,\rs,p\in\EF$,
    所以$\chi_Q(x,y,z,v)\in\EF$, 所以$Q(x,y,z,v)$是初等的.
\end{proof}
\problem{设$f:\mathbb{N}\to\mathbb{N}$满足:
$$\protect\begin{aligned}
   f(0)&=1,\\
   f(1)&=4,\\
   f(2)&=6,\\
   f(x+3)&=f(x)+(f(x+1))^2+(f(x+2))^3,
\protect\end{aligned}
$$证明: $f\in\PRF$.}
\begin{proof}
    令$g(x)=\langle f(x),f(x+1),f(x+2)\rangle$, 那么有$f(x)=\ep_0(g(x)),g(0)=\langle 1,4,6\rangle,$

    $$g(x+1)=\langle \ep_1(g(x)), \ep_2(g(x)), \ep_0(g(x))+(\ep_1(g(x)))^2+(\ep_2(g(x)))^3.$$

    令$H(x)=\langle \ep_1(x), \ep_2(x), \ep_0(x)+(\ep_1(x))^2+(\ep_2(x))^2\rangle$, 那么有
    $H(x)\in\PRF,g(x+1)=H(g(x))=H\circ P_2^2(x, g(x))$, 也就是$$g=\mathrm{Prim}^0[\langle 1,4,6\rangle, H\circ P_2^2],$$
    所以$g(x)\in\PRF,$ 所以$f(x)\in\PRF$.
\end{proof}
\problem{设$f:\mathbb{N}\to\mathbb{N}$满足:
$$\protect\begin{aligned}
    f(0)&=0,\\
    f(1)&=1,\\
    f(2)&=2^2,\\
    f(3)&=3^{3^3},\\
    \cdots&\cdots\cdots\cdots\\
    f(n)&=\protect\underbrace{n^{\cdot^{\cdot^{\cdot n}}}}_{n\textrm{个}n},
\protect\end{aligned}$$
证明:$f\in\PRF-\EF$.}
\begin{proof}
    令$$g(n,m)=\underbrace{n^{\cdot^{\cdot^{\cdot n}}}}_{m\textrm{个}n},$$
	
    且$g(0,0)=0$, 那么$g(n,m+1)=n^{g(n,m)}$, 也就是说
    $$g(n,0)=N^2(n),$$
    $$g(n,m+1)=\mathrm{Comp}^2_2[\mathrm{pow},P^3_1,P^3_3](n,m,g(n,m))$$
	
    于是$$g=\mathrm{Prim}^1[N^2,\mathrm{Comp}^2_2[\mathrm{pow},P^3_1,P^3_3]],$$
	
    得到$g\in\PRF$, 而$f(n)=g(n,n)$,故$f\in\PRF$.

    假设$f\in\EF$, 那么由定理1.34, 有
    $$\exists k\in\mathbb{N}. \forall x\in\mathbb{N}. f(x)\leqslant G(k,x).$$
	
    其中$$G(0,x)=x,G(k,x)=\underbrace{2^{\cdot^{\cdot^{\cdot 2^x}}}}_{k\textrm{个}2}.$$
	
    取$x=k+2$, 那么当$k=0$时,$f(2)>G(0,2)$, 当$k>0$时, $$f(k+2)=\underbrace{(k+2)^{\cdot^{\cdot^{\cdot (k+2)}}}}_{(k+2)\textrm{个}(k+2)},$$
    $$G(k,k+2)=\underbrace{2^{\cdot^{\cdot^{\cdot 2^{k+2}}}}}_{k\textrm{个}2},$$
	
    显然$$f(k+2)>\underbrace{(k+2)^{\cdot^{\cdot^{\cdot (k+2)}}}}_{(k+1)\textrm{个}(k+2)}>\underbrace{2^{\cdot^{\cdot^{\cdot 2^{k+2}}}}}_{k\textrm{个}2}=G(k,k+2),$$
	
    因此不存在$k$使得$\forall x\in\mathbb{N}. f(x)\leqslant G(k,x)$, 所以假设不成立, $f\notin\EF$.

    综上, $f\in\PRF-\EF$.
\end{proof}
\problem{设$g:\mathbb{N}\to\mathbb{N}\in\PRF,f:\mathbb{N}^2\to\mathbb{N}$, 满足
$$\protect\begin{aligned}
    f(x,0)&=g(x),\\
    f(x,y+1)&=f(f(\cdots f(f(x,y),y-1),\cdots),0),
   \protect\end{aligned}
$$证明:$f\in\PRF$.}
\begin{proof}
    先证明$f(x,y)=g^{2^{y-1}}(x)$当$y\geqslant 1$时成立.

    当$y=1$时, $f(x,y)=f(x,0)=g(x)$, 成立;

    假设当$y\leqslant k$时命题成立, 那么有$f(x,y)=g^{2^{y-1}}(x)$在$1\leqslant y\leqslant k$时对任意的$x\in\mathbb{N}$均成立.

    当$y=k+1$时, $$\begin{aligned}
        f(x,k+1)&=f(f(\cdots f(f(x,k),k-1),\cdots),0)\\
        &=f(f(\cdots f(g^{2^{k-1}}(x),k-1),\cdots), 0)\\
        &=f(f(\cdots g^{2^{k-2}}(g^{2^{k-1}}(x)),\cdots), 0)\\
        &=f(g^{2^0}\circ g^{2^1}\circ \cdots \circ g^{2^{k-2}}\circ g^{2^{k-1}}(x),0)\\
        &=g(g^{\frac{1-2^k}{1-2}}(x))\\
        &=g^{2^k}(x)
    \end{aligned}$$
	
    由数学归纳法可知, $f(x,y)=g^{2^{y-1}}(x)$.
    
    现在只要证明$g^{2^{y-1}}(x)\in\PRF$.

    由于$g(x)\in\PRF$, 所以$G(x,n)=g^n(x)\in\PRF$(由定义1.27可知$G(x,n)\in\mathcal{ITF}$, 而$\mathcal{ITF}=\PRF$). 所以只要证明$$F(y)=\begin{cases}
        2^{y-1},&y>0\\
        1,&y=0,
    \end{cases}$$
	
	$F(y)\in\PRF$即可.
    而$$F(y)=\left\lfloor\frac{2^yN^2(y)+2N(y)}{2}\right\rfloor,$$
	
	所以$F(y)\in\PRF$.
    综上, $f\in\PRF$.
\end{proof}
\problem{设$k\in\mathbb{N}^+$, 函数$f:\mathbb{N}^k\to\mathbb{N}$和$g:\mathbb{N}^k\to\mathbb{N}$仅在有穷个点取不同值, 证明: $f$为递归函数当且仅当$g$为递归函数.}
\begin{proof}
    根据对称性, 只要证明$f$为递归函数可以推出$g$为递归函数即可.

    由于$f$和$g$仅在有穷个点取不同值, 这些取不同值的点构成一个有穷的集合, 记为$\mathcal{S}$, 这样, $f(\vec{x})\neq g(\vec{x})$当且仅当$\vec{x}\in\mathcal{S}$, $f(\vec{x})=g(\vec{x})$当且仅当$\vec{x}\in\mathbb{N}^k-\mathcal{S}$.

    设$\mathcal{S}$有$k$个元素, 那么其可表示为$\mathcal{S}=\{\vec{x}_1,\vec{x}_2,\cdots,\vec{x}_k\}$, 那么, 有
    $$g(\vec{x})=f(\vec{x})(\prod_{i=1}^k \eq(\vec{x},\vec{x}_i))+\sum_{i=1}^k N(\eq(\vec{x},\vec{x}_i))g(\vec{x}_i)$$
    上式中,$\eq$被扩展为$\mathbb{N}^a\times\mathbb{N}^a\to\mathbb{N}$的函数, 这不影响其依然是初等函数, 因为可以表示成
	$$N^2(\sum_{i=1}^a \eq(P^a_i(\vec{x}), P^a_i(\vec{x}_i))).$$

	因此, 若$f\in\GRF$, 即可得到$g\in\GRF$.
\end{proof}
\problem{证明:$$f(n)=\left\lfloor\left(\frac{\sqrt{5}+1}{2}\right)n\right\rfloor$$为初等函数.}
\begin{proof}
    因为$\frac{\sqrt{5}+1}{2}\leqslant 2$, 所以$f(n)\leqslant 2n$, 也就是说
    $$f(n)=\max x\leqslant 2n. \left[x\leqslant \left(\frac{\sqrt{5}+1}{2}\right)n\right]$$
	
    即在$[0,2n]$内不大于$\left(\frac{\sqrt{5}+1}{2}\right)$的最大自然数.
    $$\begin{aligned}
        x&\leqslant \left(\frac{\sqrt{5}+1}{2}\right)n\\
        \Leftrightarrow &2x\leqslant (\sqrt{5}+1)n\\
        \Leftrightarrow &2x-n\leqslant \sqrt{5}n\\
        \Leftrightarrow &4x^2-4xn+n^2\leqslant 5n^2\\
        \Leftrightarrow &x^2-xn-n^2\leq 0\\
        \Leftrightarrow &x^2\dotdiv xn\dotdiv n^2 = 0
    \end{aligned}$$
	
    所以$f(n)=\max x\leqslant 2n.[N^2(x^2\dotdiv xn\dotdiv n^2)]\in\EF$.
\end{proof}
\iffalse
\subsubsection{$f(n)=\left\lfloor\left(\frac{\sqrt{5}+1}{2}\right)^n\right\rfloor\in\EF.$}
    依照宋公的说法, 题目本来是$f(n)=\left\lfloor\left(\frac{\sqrt{5}+1}{2}\right)^n\right\rfloor$的, 以下给出证明:
\begin{proof}
    因为$\frac{\sqrt{5}+1}{2}\leqslant 2$, 所以$f(n)\leqslant 2^n$, 也就是说
    $$f(n)=\max x\leqslant 2^n. \left[x\leqslant \left(\frac{\sqrt{5}+1}{2}\right)^n\right]$$
    $$\begin{aligned}
        x&\leqslant \left(\frac{\sqrt{5}+1}{2}\right)^n\\
        \Leftrightarrow &2^nx\leqslant (\sqrt{5}+1)^n\\
        \Leftrightarrow &2^{2n}x^2\leqslant (\sqrt{5}+1)^{2n}\\
        \Leftrightarrow &2^{2n}x^2\leqslant \sum_{i=0}^{2n}{2n \choose i}(\sqrt{5})^{2n-1}
    \end{aligned}$$
    而
    $$\sum_{i=0}^{2n}{2n \choose i}(\sqrt{5})^{2n-1}=\sum_{i=0}^{n} {2n \choose 2i} 5^{n-i}+\sum_{i=0}^{n-1}{2n \choose 2i+1}5^{n-i-1}\cdot \sqrt{5}$$
    令$g_1(n)=\sum_{i=0}^{n} {2n \choose 2i} 5^{n-i}$, $g_2(n)=\sum_{i=0}^{n-1}{2n \choose 2i+1}5^{n-i-1}$, 易证$g_1,g_2\in\EF$.

    故需要证明$2^{2n}x^2\leqslant g_1(n)+g_2(n)\cdot\sqrt{5}$是初等数论谓词.
    $$\begin{aligned}
        2^{2n}x^2&\leqslant g_1(n)+g_2(n)\cdot\sqrt{5}\\
        \Rightarrow (2^{2n}x^2-g_1(n))^2\leqslant (g_1(n))^2\cdot 5
    \end{aligned}$$
\end{proof}
\fi
\problem{证明:$\mathrm{Ack}(4,n)\in\PRF-\EF$, 其中$\mathrm{Ack}(x,y)$是Ackermann函数.}
\begin{proof}
    记$f(n)=\mathrm{Ack}(4,n)+3=\underbrace{2^{\cdot^{\cdot^{\cdot 2}}}}_{n+3\text{个}2}$.

    令$g(x,y)=2^y$, 那么有$f(0)=2^{2^2},f(n+1)=g(n,f(n))$, 即$f=\mathrm{Prim}^0[2^{2^2},g]$,
    因为$g\in\PRF$, 所以$f\in\PRF$, 自然$\mathrm{Ack}(4,n)\in\PRF$.

    假设$f\in\EF$, 那么对于$\EF$的控制函数$G(k,x)=\underbrace{2^{2^{\cdot^{\cdot^{\cdot 2^x}}}}}_{k\text{个}2}$, 存在$k\in\mathbb{N}$使得$$\forall x\in\mathbb{N}.f(x)\leqslant G(k,x)$$.

    然而, 令$x=2k$, 就有
    $$f(2k)=\underbrace{2^{\cdot^{\cdot^{\cdot 2}}}}_{(2k+3)\text{个}2}=G(k,\underbrace{2^{\cdot^{\cdot^{\cdot 2}}}}_{k+3\text{个}2})>G(k,2k)=G(k,x),$$
    矛盾, 故$f\notin\EF$.

    而且, $\underbrace{2^{\cdot^{\cdot^{\cdot 2}}}}_{(2k+3)\text{个}2}-\underbrace{2^{\cdot^{\cdot^{\cdot 2k}}}}_{(k+1)\text{个}2}$在$k=0$时最小, 此时的值也大于3, 因此$\mathrm{Ack}(4,n)\notin\EF$.

    综上, $\mathrm{Ack}(4,n)\in\PRF-\EF$.
\end{proof}
\problem{设$f:\mathbb{N}\to\mathbb{N}$,$f$为单射(1-1)且满射(onto), 证明:$f\in\GRF\Leftrightarrow f^{-1}\in\GRF$.}
\begin{proof}
    $f^{-1}(y)$的值是使得$f(x)=y$的$x$值, 因为$f$为单射, 这样的$x$值是唯一的, 因此
    $$f^{-1}(y)=\mu x.[f(x)\ddot{-}y]$$

    由于$f$是满射, 因此$\forall y\in\mathbb{N}. \exists x\in\mathbb{N}. f(x)\ddot{-}y=0$,
    所以$F(x,y)=f(x)\ddot{-}y$是正则的.

    $\GRF$对正则$\mu$-算子封闭, 因此$f\in\GRF\Rightarrow f^{-1}\in\GRF$.

    因为$f$为单射且满射, 所以$f^{-1}$也是单射且满射, 且$(f^{-1})^{-1}=f$, 所以$f^{-1}\in\GRF\Rightarrow f\in\GRF$.

    综上, $f\in\GRF\Leftrightarrow f^{-1}\in\GRF$.
\end{proof}
\problem{设$p(x)$为整系数多项式, 令$f:\mathbb{N}\to\mathbb{N}$定义为$f(a)=p(x)-a$对于$x$的非负整数根, 证明: $f\in\RF$.}
\begin{proof}
    令$p(x)=a_nx^n+\cdots+a_1x+a_0, \mathcal{P}=\{i|a_i>0\}, \mathcal{N}=\{i|a_i<0\}$, 有
    $$|p(x)-a|=\sum_{i\in\mathcal{P}} |a_i|x^i \ddot{-} \left(a+\sum_{i\in\mathcal{N}} |a_i|x^i\right)$$
	
    此时$\ddot{-}$两边的式子都是正整系数多项式, 因此$|p(x)-a|\in\EF$.

    而$f=\mu x.[|p(x)-a|]$, 即$f(a)$是让$|p(x)-a|=0$的最小解\footnote{题目中只是说明了$f(a)$对于$x$的非负整数根, 但$f(a)$的非负整数根可能不止一个, 此题可能有些问题.}($|p(x)-a|$处处有定义).

    所以$f\in\RF$.
\end{proof}
\problem{设
$$f(x)=\protect\begin{cases}
    x/y,&\textrm{若}y\neq 0\textrm{且}y\mid x,\\
    \uparrow,&\textrm{否则.}
\protect\end{cases}$$证明:$f\in\RF$.}
\begin{proof}
    可以看出,$f(x)$是使得$ay=x$成立的$a$, 且要求$y\neq 0$时才有定义, 故
    $$f(x)=\mu a.[(x\ddot{-}ay)+N(y)]$$
	
    所以$f\in\RF$.
\end{proof}
\problem{设$g:\mathbb{N}\to\mathbb{N}$满足:
$$\protect\begin{aligned}
    g(0)&=0,\\
    g(1)&=1,\\
    g(n+2)&=\rs((2002g(n+1)+2003g(n)),2005),
\protect\end{aligned}
$$(1) 试求$g(2006)$, (2) 证明$g\in\EF$.}
\begin{proof}
    先考虑这么一个数列:$a_1=0,a_2=1,a_{n+2}=2002a_{n+1}+2003a_{n}$.

    我们希望把上面的递推式改写成$a_{n+2}-ba_{n+1}=c(a_{n+1}-ba_n)$的形式, 那么需要得到合适的$b,c$使得$b+c=2002,-bc=2003$.

    根据韦达定理\footnote{一元二次方程$ax^2+bx+c=0(a\neq 0)$的两个根$x_1,x_2$满足$x_1+x_2=-\frac{b}{a}, x_1x_2=\frac{c}{a}$.}, 上式中的$b,c$是方程$x^2-2002x+2003=0$的两个根, 解得其两个根为-1和2003.

    有$a_{n+2}+a_{n+1}=2003(a_{n+1}+a_n)$且$a_{n+2}-2003a_{n+1}=-(a_{n+1}-2003a_n)$,
    那么有$a_{n+1}+a_{n}=2003^{n-1}(a_2+a_1)=2003^{n-1}$且$a_{n+1}-2003a_n=(-1)^{n-1}(a_2-2003a_1)=(-1)^{n-1}$, 联立解得:
    $$a_n=\frac{2003^{n-1}-(-1)^{n-1}}{2004}$$
    
    \textbf{现在证明}$g(n)\equiv a_{n+1}\pmod{2005}$.
    
    当$n=0,1$时, 显然成立.

    假设当$n\leqslant k(k\geqslant 1)$时命题成立, 那么$g(k)\equiv a_{k+1}\pmod{2005}, g(k-1)\equiv a_{k}\pmod{2005}$.

    当$n=k+1$时, $g(k+1)=\rs((2002g(k)+2003g(k-1)),2005)$, 显然$g(k+1)\equiv (2002g(k)+2003g(k-1))\pmod{2005}$.

    由于$g(k)\equiv a_{k+1}\pmod{2005}$, 可得$2002g(k)\equiv 2002a_{k+1}\pmod{2005}$.\footnote{$a\equiv b\pmod{m}\Rightarrow an\equiv bn\pmod{m}, \forall n\in \mathbb{Z}.$}

    由于$g(k-1)\equiv a_{k}\pmod{2005}$, 可得$2003g(k-1)\equiv 2003a_{k}\pmod{2005}$.

    所以$2002g(k)+2003g(k-1)\equiv 2002a_{k+1}+2003a_k\pmod{2005}$.\footnote{$a\equiv b\pmod{m},c\equiv d\pmod{m}\Rightarrow a\pm c\equiv b\pm d\pmod{m}.$}

    所以$g(k+1)\equiv 2002a_{k+1}+2003a_k\pmod{2005}$,\footnote{$a\equiv b\pmod{m},b\equiv c\pmod{m}\Rightarrow a\equiv c\pmod{m}.$} 即$g(k+1)\equiv a_{k+2}\pmod{2005}$成立.

    由数学归纳法得, 命题成立.

    故$$\begin{aligned}
        g(n)&=\rs(a_{n+1},2005)\\
        &=\rs(\frac{2003^n-(-1)^n}{2004},2005)\\
        &=\rs(\frac{2003^n+1}{2004}\rs(n,2)+\frac{2003^n-1}{2004}N(\rs(n,2)),2005),
    \end{aligned}$$ 所以$g\in\EF$.

    现在开始求$g(2006)$.

    首先$g(2006)=\rs(\frac{2003^{2006}-1}{2004}, 2005)$, 即有$\frac{2003^{2006}-1}{2004}\equiv g(2006)\pmod{2005}$.

    因为$2005=5\cdot 401$, 5和401均为素数, 根据费马小定理\footnote{若$a$是一个整数, $p$是一个素数, 且$a$不是$p$的倍数, 则有$a^{p-1}\equiv 1\pmod{p}.$}, 有$2^4\equiv 1\pmod{5},2^{400}\equiv 1\pmod{401}$. 我们还有$2^{400}\equiv 1\pmod{5}$(令$a=2^{100}$由费马小定理得到). 而401和5的最小公倍数为2005, 故$2^{400}\equiv 1\pmod{2005}$.\footnote{设$m_1,m_2,\cdots,m_n$的最小公倍数为$[m_1,m_2,\cdots,m_n]$, 且$\forall i=1,2,\cdots,n. a\equiv b\pmod{m_i}$, 则有$a\equiv b\pmod{[m_1,m_2,\cdots,m_n]}$}.

    因为$m\mid(m-a)^2-a^2$, 所以$(m-a)^2 \equiv a^2\pmod{m}$. 也就是说, $2003^2\equiv 2^2\pmod{2005},2004^2\equiv 1^2\pmod{2005}$. 那么$2003^{2006}\equiv 2^{2006}\pmod{2005}$.\footnote{$a\equiv b\pmod{m}\Rightarrow a^n\equiv b^n\pmod{m}$.} 

    那么, $\frac{2003^{2006}-1}{2004}\cdot 2004^2 \equiv g(2006)\cdot 1\pmod{2005}$.\footnote{$a\equiv b\pmod{m},c\equiv d\pmod{m}\Rightarrow ac\equiv bd\pmod{m}.$}即$(2003^{2006}-1)\cdot 2004\equiv g(2006)\pmod{2005}$.

    而$(2003^{2006}-1)\cdot 2004\equiv (2^{2006}-1)\cdot 2004\pmod{2005}$.

    又因为$2^{400}\equiv 1\pmod{2005}$, 所以$2^{2000}\equiv 1\pmod{2005}, 2^{2006}\equiv 2^6\pmod{2005}$. 所以$(2^{2006}-1)\cdot 2004\equiv (2^6-1)\cdot 2004 \pmod{2005}$. 于是我们得到$g(2006)\equiv (2^6-1)\cdot 2004\pmod{2005}$, 只要计算$(2^6-1)\cdot 2004$除以2005的余数即可得到$g(2006)$.

    而$2004\equiv -1\pmod{2005}$, 所以$63\cdot 2004\equiv -63\pmod{2005}$, 所以$63\cdot 2004\equiv -63+2005\pmod{2005}$, 所以$g(2006)= 2005 - 63 = 1942$.

\end{proof}
\problem{设$f:\mathbb{N}\to\mathbb{N}$定义为$$f(n)=\pi\text{的十进制展开式中第}n\text{位数字}.$$
例如$f(0)=3,f(1)=1,f(2)=4$. 证明: $f\in\GRF$.}
\begin{proof}
    由Hutton's Formula\footnote{实际上, 类似Hutton's Formula的公式还有许多, 应用最广泛的是梅钦公式(Machin's Formula), 这一类公式因此也被称为类梅钦公式(Machin-like Formulas), 详见\texttt{http://mathworld.wolfram.com/Machin-LikeFormulas.html}}知, $\frac{\pi}{4}=2\arctan(\frac13)+\arctan(\frac17)$.

    而$$\arctan x=\int_0^x\frac{1}{1+t^2}\mathrm{d}t,$$

    $\frac{1}{1+t^2}$应用牛顿广义二项式定理\footnote{$(x+y)^{\alpha }=\sum _{{k=0}}^{\infty }\binom{\alpha}{k}x^{{\alpha -k}}y^{k}. $其中$\binom{\alpha}{k}=\frac{\alpha(\alpha-1)\cdots(\alpha-k+1)}{k!}$}展开, $k=n$之后的项利用等比数列求和公式, 得到: $$\frac{1}{1+t^2}=\sum_{i=0}^n[(-1)^ix^{2i}]+\frac{(-1)^{n+1}x^{2n+2}}{1+x^2}$$

    因此\begin{align*}
    \arctan x &=\sum_{i=0}^n (-1)^i \int_0^xt^{2i}\mathrm{d}t+\int_0^x\frac{(-1)^{n+1}}{1+t^2}t^{2n+2}\mathrm{d}t\\
    &=\sum_{i=0}^n(-1)^i\frac{x^{2i+1}}{2i+1}+\int_0^x\frac{(-1)^{n+1}}{1+t^2}t^{2n+2}\mathrm{d}t\tag{*}
    \end{align*}

    为了让余项为正且估计更加精确, 在(*)中取$n=2k+1$.

    \begin{align*}
        \pi &= 8\arctan(\frac{1}{3})+4\arctan(\frac{1}{7})\\
        &=8\sum_{i=0}^{2k+1}(-1)^i\frac{1}{(2i+1)3^{2i+1}}+8\int_0^{\frac{1}{3}}\frac{t^{4k+4}}{1+t^2}\mathrm{d}t\\
        &+4\sum_{i=0}^{2k+1}(-1)^i\frac{1}{(2i+1)7^{2i+1}}+4\int_0^{\frac{1}{7}}\frac{t^{4k+4}}{1+t^2}\mathrm{d}t
    \end{align*}

    令
    \begin{align*}
        t_k &=8\sum_{i=0}^{2k+1}(-1)^i\frac{1}{(2i+1)3^{2i+1}}+4\sum_{i=0}^{2k+1}(-1)^i\frac{1}{(2i+1)7^{2i+1}}\\
        r_k &=8\int_0^{\frac{1}{3}}\frac{t^{4k+4}}{1+t^2}\mathrm{d}t+4\int_0^{\frac{1}{7}}\frac{t^{4k+4}}{1+t^2}\mathrm{d}t\\
        &\leqslant8\int_0^{\frac{1}{3}}t^{4k+4}\mathrm{d}t+4\int_0^{\frac{1}{7}}t^{4k+4}\mathrm{d}t\\
        &=8\cdot \frac{1}{3^{4k+5}\cdot (4k+5)}+4\cdot\frac{1}{7^{4k+5}\cdot(4k+5)}\\
        &< 8\cdot\frac13\cdot\frac{1}{3^{4k+4}}+4\cdot\frac17\cdot\frac{1}{7^{4k+4}}\leqslant \frac{1}{3^{4k}}\leqslant \frac{1}{80^k}
    \end{align*}

    于是$\pi=t_k+r_k$, 且$0<r_k<\frac{1}{80^k}$. $t_k$无论$k$取多少都为有理数. 这样的话, 我们取$t_k$的第$k$位为$\pi$的第$k$位时, 由于$10^kt_k<10^k\pi=10^k(t_k+r_k)<10^kt_k+\frac{1}{8^k}\leqslant 10^kt_k+1$, 这样的误差会使得当$t_k$的第$k$位为9以外的数时, 其前面的结果都是准确无误的. (比如, 当我们知道$3.14<\pi<3.15$时, 我们可以保证3.1是正确的. 但是思考0.9999...的情形, 它的确实结果是1, 从第0位开始就错了)

    现在要求当$t_k$的第$k$位为9时, 其前面的结果也准确无误. 可以发现, 如果此时$t_{k+1}$的第$k+1$位非9时, 由于精度更进了一层, 所以$\pi$的第$k$位就必然为9. 由数学归纳法可知, 当$t_{k}$的第$k$位非9时, 必然能保证其估计的$\pi$值的前$k$位均正确. 但是, 如果从第$l$位出现连续的9, 那么就无法确定第$l-1$位是$t_{l-1}$的第$l+1$位还是它的值加1. 设$t_k$的十进制展开式为
    $$t_k=a_{k,0}a_{k,1}a_{k,2}\cdots a_{k,n}\cdots$$

    现在需要证明, 对于任意的$n\in\mathbb{N}$, 存在$l\geqslant n+1$使得$t_l$满足以下数字不全为9: $a_{l,n+1},a_{l,n+2},\cdots,a_{l,l}$. 这样做就能保证我们那个非9所在位之前的数均是正确的.

    假设这样的$l$不存在, 那么这意味着从第$n+1$位起, 后面的数字均为9, 出现了循环, 那么这就说明$\pi$是有理数, 矛盾. 因此上面的命题是正确的.

    这样还能带来一个好处, 就是取第$k$位时, 第$k$位也是正确的, 因为0.9999...的情形不用再考虑了, 当知道$2.9<a<3$时, 我们也能确定$a$展开式以2.9开头.

    令$l=l(n)=\mu l.[l\geqslant n+1$且在$a_{l,n+1},\cdots,a_{l,l}$并不全为9$]$,$a(l,i)=a_{l,i}=t_l$展开式的第$i$个数字$=\rs(\lfloor t_l\cdot 10^i\rfloor,10)\in\EF$.

    现在能保证$t_l$展开式中$l$所在位及其之前的数均正确, 即$f(n)=a(l,n)$. 而
    $$l(n)=\mu l.\left[(l\geqslant n+1)\land \prod_{i=n+1}^l[a_{l,i}\ddot{-}9\neq 0]\right]\in\GRF.\footnote{若知道$\pi$的展开式中连续9的个数有上界, 则$l(n)\in\EF$.}$$

    因此
    $$f(n)=a(l(n),n)\in\GRF.$$
\end{proof}
\newpage
% Chapter 2
\section{算盘机}
\problem{构造AM计算函数$f(x)=2x$.}
\begin{solution}
  $f=\L S_1A_2A_3\R_1\mv_{2,1}\mv_{3,1}$
\end{solution}
\problem{构造AM计算函数$f(x)=\lfloor x/2\rfloor$.}
\begin{solution}
$f=A_1\L S_1S_1A_2\R_1\mv_{2,1}S_1$.
\end{solution}
\problem{构造AM计算函数$f(x,y)=x\cdot y$.}
\begin{solution}
$f=\mv_{1,3}\L\cp_{3,1,4}S_2\R_2Z_3$
\end{solution}
\problem{构造AM计算函数$f(x)=2^x$.}
\begin{solution}
$f=\mv_{1,2}A_1\L\cp_{1,3}\mv_{3,1}S_2\R_2$.
\end{solution}
\problem{设$f:\mathbb{N}\to\mathbb{N}$且$F\in$AM计算函数$f$, 构造算盘机$G$使得
$$(0,0,0,\cdots,0,\cdots)G=
\protect\begin{cases}
    (a,0,0,\cdots)&,\text{若}a\text{为}f\text{的最小根},\\
    \uparrow&,\text{若}f\text{无根}.
\protect\end{cases}$$}
\begin{solution}
思路如下:
一个档专门表示自变量$x$, 假设$f$需要再额外用到$k$个档, 先计算$f(0)$, 若为0, 不进行操作. $k+2$档即$x$自增1, 清除1档结果后复制到1档, 计算$f(x)$, 如果1档(即$f$的值)的数不为0则重复, 到达0后,  移动到1档, 完成. 如果没有根, 这个算盘机将永远算下去.

若$f$需要$k$个档, 这个算盘机可表示为$$f\L A_{k+2}Z_1\cp_{k+2,1,k+3}f\R_1\mv_{k+2,1}.$$
\end{solution}
\newpage
% Chapter 3
\section{$\lambda$-演算}
\problem{证明括号引理: 对于任何$M\in \Lambda$, 在$M$中出现的左括号的个数等于在$M$中出现的右括号的个数.}
\begin{proof}
    令$L(M)$为在$M$中出现的左括号的个数, $R(M)$为在$M$中出现的右括号的个数. 现在只要证明: 对于任何$M\in\Lambda$, $L(M)=R(M)$.

    当$M\in\nabla$时, $L(M)=R(M)=0$, 命题成立.

    假设当$M,N\in\Lambda$时, $L(M)=R(M)$, $L(N)=R(N)$, 那么令$C=(MN)$, 有$L(C)=L(M)+L(N)+1$, $R(C)=R(M)+R(N)+1$, 可得$L(C)=R(C)$.

    假设当$M\in\Lambda, x\in\nabla$时, $L(M)=R(M)$, 那么令$F=(\lambda x.M)$, 有$L(F)=L(M)+1, R(F)=R(M)+1$, 可得$L(F)=R(F)$.

    由结构归纳法可得, 命题成立.
\end{proof}
\problem{试求$SSSS$的$\bnf$.}
\begin{solution}
    根据定义, $S\equiv \lambda xyz.xz(yz)$, 那么
    \begin{align*}
        SS&\equivbeta \lambda yz.Sz(yz)\\
        &\equivbeta \lambda yz.\lambda f.zf((yz)f)\\
        &\equiv \lambda yzf.zf(yzf)\\
        &\equiv \lambda xyz.yz(xyz)
    \end{align*}

    而
    \begin{align*}
        SSSS &\equivbeta SS(SS)\\
        &\equivbeta \lambda yz.yz(SSyz)\\
        &\equivbeta \lambda yz.yz(\lambda f.zf(yzf))\\
        &\equiv \lambda xy.xy(\lambda z.yz(xyz))\tag{*}
    \end{align*}

    到了(*)式后, 已经没有含有$\beta$-可约式的子项, 故(*)式即为$SSSS$的$\bnf$.
\end{solution}
\problem{证明: $(\lambda x.xxx)(\lambda x.xxx)$没有$\bnf$.}
\begin{proof}
	\begin{align*}
		(\lambda x.xxx)(\lambda x.xxx)&\onestepbeta (\lambda x.xxx)(\lambda x.xxx)(\lambda x.xxx)\\
		&\onestepbeta (\lambda x.xxx)(\lambda x.xxx)(\lambda x.xxx)\\
		&\onestepbeta (\lambda x.xxx)(\lambda x.xxx)(\lambda x.xxx)(\lambda x.xxx)
	\end{align*}
	
	可以看出, 若令$(\lambda x.xxx)=N$, 记$\underbrace{NN\cdots N}_{k\text{个}N}$为$N^k$, 那么可以发现对任意$k>1$, $N^k\onestepbeta N^{k+1}$, 因此每次做$\beta$-归约一次, 得到的式子都包含$NN$这个有$\beta$-可约式, 永远都无法归约得到$\bnf$. 所以$(\lambda x.xxx)(\lambda x.xxx)$没有$\bnf$.
\end{proof}
\problem{设$F\in\Lambda$呈形$\lambda x.M$, 证明:\\
    (1) $\lambda z.Fz\equivbeta F$;\\
    (2) $\lambda z.yz\neq_\beta y$. \\
注意, 对于一般的$F$, $\lambda z.Fz\neq_\beta F$, 但$\lambda z.Fz=_{\eta} F$.}
\begin{proof}
    $\lambda z.Fz\equiv \lambda z.(\lambda x.M)z\equivbeta \lambda z.M[x:=z]\equiv M$.
	
	$\lambda z.yz$本身已是$\bnf$, 其无法$\beta$-归约到$y$.
\end{proof}
\problem{证明二元不动点定理: 对于任何$F,G\in\Lambda$, 存在$X,Y\in\Lambda$, 满足$$FXY=X,$$ $$GXY=Y.$$}
\begin{proof}
    首先, 根据不动点定理, 存在$Y$使得$GXY=Y$, 这可以表示为$Y=\mathbf{Y}(GX)$, $\mathbf{Y}$为不动点组合子.
	
	那么现在只要证明, 对于任何$F,G\in\Lambda$, 存在$X\in\Lambda$, 满足$FX(\mathbf{Y}(GX))=X$.
	
	而 $FX(\mathbf{Y}(GX))=X\Rightarrow (\lambda x.Fx(\mathbf{Y}(Gx)))X=X$ (根据($\beta),(\sigma),(\tau$)), 于是$$X=\mathbf{Y}(\lambda x.Fx\mathbf{Y}(Gx)).$$
	
	也就是说, 令
	
	\begin{align*}
		X=&\mathbf{Y}(\lambda x.Fx(\mathbf{Y}(Gx))),\\
		Y=&\mathbf{Y}(G\mathbf{Y}(\lambda x.Fx(\mathbf{Y}(Gx))))
	\end{align*}
	
	就可以让任何$F,G\in\Lambda$满足$FXY=X, GXY=Y$.
\end{proof}
\problem{证明: 对任何$M,N\in\Lambdaclosed$, 方程$xN=Mx$对于$x$有解.}
\begin{proof}
    令$x$呈形$\lambda a. B$, 那么有$xN=B$成立, 问题可转化为$B=M\lambda a. B$对$B$有解.
	$$B=M(\lambda a. B)\Rightarrow B=(\lambda b.M(\lambda a. b)) B.$$
	
	令$\mathbf{Y}$为不动点组合子, 那么$B=\mathbf{Y}(\lambda b.M(\lambda a.b))$.
	
	因此$x$有解, $x=\lambda c.\mathbf{Y}(\lambda b.M(\lambda a.b))$可使等式$xN=Mx$成立.
\end{proof}
\problem{证明: 对任何$P,Q\in\Lambda$, 若$P\manystepbeta Q$, 则存在$n\geq 0$以及$P_0,\cdots,P_n\in\Lambda$,满足\\
(1) $P\equiv P_0$;\\
(2) $Q\equiv P_n$;\\
(3) 对任何$i<n, P_i\onestepbeta P_{i+1}$.}
\begin{proof}
    由于关系$\manystepbeta$是关系$\onestepbeta$的自反传递闭包, 因此对于所有的$P,Q\in\Lambda$, 
	$$\manystepbeta=\bigcup_{i\in\mathbb{N}} (\onestepbeta)^i.$$
	
	因此, $P\manystepbeta Q$可以等价地表述为: 存在$n\in\mathbb{N}$, 使得$P(\onestepbeta)^n Q$.
	
	因此, 对任意$i=1,2,\cdots, n-1$, 令$P(\onestepbeta)^iP_i$, 即可满足题中所给的3个条件.
\end{proof}
\problem{证明: 对于任意$P,Q\in\Lambda$, 若$P\manystepbeta Q$, 则$\lambda z.P\manystepbeta \lambda z.Q$.}
\begin{proof}
    根据题3.7的结论, $P\manystepbeta Q$可得到存在$P_0,\cdots,P_n\in\Lambda$使得$P\equiv P_0,Q\equiv P_n, P_i\equiv P_{i+1}$.
	
	由于$\onestepbeta$是$\beta$的合拍闭包, 因此可得对任意$A, B\in\Lambda$, 若$A\onestepbeta B$, 则$\lambda x.A\onestepbeta\lambda x.B$. 于是有$\lambda z.P_i\onestepbeta\lambda z.P_{i+1}$对任意自然数$i, i<n$都成立.
	
	因此, $\lambda z.P\manystepbeta\lambda z.Q$.
\end{proof}
\problem{证明: 对于任意$P,Q\in\Lambda$, 若$P\equivbeta Q$, 则存在$n\in\mathbb{N}$以及$P_0,\cdots,P_n\in\Lambda$,满足\\
(1) $P\equiv P_0$;\\
(2) $Q\equiv P_n$;\\
(3) 对任何$i<n$, $P_i\onestepbeta P_{i+1}$或$P_{i+1}\onestepbeta P_i$.}
\begin{proof}
    由于$=_\beta$是$\manystepbeta$的对称闭包, 因此$=_\beta=\{\manystepbeta, \manystepbeta^{-1}\}$, 后者定义为, $A,B\in\Lambda, A\manystepbeta B$当且仅当$B\manystepbeta^{-1}A$.
	
	再由题3.7, 可得
	$$=_\beta=\bigcup_{i\in\mathbb{N}}(\onestepbeta\cup\onestepbeta^{-1})^i.$$
	
	因此, 对任意$i=1,2,\cdots, n-1$, 令$P(=_\beta)^iP_i$, 即可满足题中所给的3个条件.
\end{proof}
\problem{证明定理3.12.}
定理3.12表述为:

对任何$M,N\in\Lambda$,$$M\equivbeta N\Leftrightarrow \lambda\beta\vdash M=N.$$
\begin{proof}
    $\lambda\beta$的公理表明$M=M$, $(\lambda x.M)N=M[x:=N]$. 
	
	由于$\beta\equiv\{((\lambda x.M)N, M[x:=N]):M,N\in\Lambda\land x\in \nabla\}$, 故$(\lambda x.M)N\equivbeta M[x:=N]$.
	
	由于$=_{\beta}$是自反的, 故$M\equivbeta M$.
	
	假设对于所有构造长度不大于$\ell$的公式$M=N$都有$\lambda\beta\vdash M=N\Rightarrow M\equivbeta N$. 那么对于构造长度为$\ell+1$的公式, 有:
	
	\begin{enumerate}
		\item $(\sigma): M=N\vdash N=M$, 由于$\equivbeta$是对称的, 因此$N\equivbeta M$成立.
		\item $(\tau): M=N,N=L\vdash M=L$, 由于$\equivbeta$是传递的, 因此$M\equivbeta L$成立.
		\item $(\mu): M=N\vdash ZM=ZN$, 由于$\equivbeta$是合拍的, 因此$ZM\equivbeta ZN$也成立.
		\item $(\nu): M=N\vdash MZ=NZ$, 由于$\equivbeta$是合拍的, 因此$MZ\equivbeta NZ$也成立.
		\item $(\xi): M=N\vdash \lambda x.M=\lambda x.N$, 由于$\equivbeta$是合拍的, 因此$\lambda x.M=\lambda x.N$也成立.
	\end{enumerate}
	
	因此, $\lambda\beta\vdash M=N\Rightarrow M\equivbeta N$.
	
	现在证明$M\equivbeta N\Rightarrow \lambda\beta\vdash M=N$.
	
	$M\equivbeta N$要么满足$(M,N)\in\beta$, 这时由二元关系$\beta$的定义知$M=N$; 要么$(M,N)$在${(M',N')}$的合拍闭包中, 其中$(M',N')\in\beta$.
	
	由题3.9可知, 对所有$M\equivbeta N$, 存在$n\in\mathbb{N}$以及$P_0,\cdots,P_n\in\Lambda$,满足$M\equiv P_0, N\equiv P_n$, 对任何$i<n$, $P_i\onestepbeta P_{i+1}$或$P_{i+1}\onestepbeta P_i$.
	
	当$n=0$时, $M\equivbeta M\Rightarrow \lambda\beta\vdash M=M$由自反性知显然成立.
	
	那么当$n=k$时, 我们假设构造长度为$m(m<n)$时, 所有$A\equivbeta B\Rightarrow \lambda\beta\vdash M=M$. 这样的话, 对于构造序列$M=P_0,\cdots,P_{n-1},P_n=N$, 有$\lambda\beta\vdash M=P_{n-1}$. 因为$P_{n-1}\onestepbeta P_n$或$P_n\onestepbeta P_{n-1}$就是说$\lambda\beta\vdash P_{n-1}=P_n$, 由$(\tau)$, $\lambda\beta\vdash M=P_{n-1}=P_n=N$. 所以$M\equivbeta N\Rightarrow \lambda\beta\vdash M=N$.
	
	综上, 命题成立.
\end{proof}
\problem{证明定理3.13.}
定理3.13表述为:

对任何$M,N\in\Lambda$,$$M=_{\beta\eta} N\Leftrightarrow \lambda\beta\eta\vdash M=N.$$
\begin{proof}
    在题3.10的基础上, 只要证明$\lambda x.Mx=_{\beta\eta} M$即可. 而这从二元关系$\eta$的定义就可以直接得出.
\end{proof}
\problem{证明: 对于任何$M,N\in\Lambda$, 若$M\equivbeta N$, 则存在$T$使$M\manystepbeta T$且$N\manystepbeta T$. 这就是对于$\equivbeta$的CR性质.}
\begin{proof}
    $M\equivbeta N$蕴含
	$$(M,N)\in\bigcup_{i\in\mathbb{N}}(\to_{\beta}\cup\leftarrow_{\beta})^k.$$
	
	当$k=0$时, $M\equivbeta N$.
	假设对所有$(M,N)\in(\to_{\beta}\cup\leftarrow_{\beta})^k$, 存在$T\in\Lambda$使得$M\manystepbeta T$且$N\manystepbeta T$.
	
	那么当$(M,N)\in(\to_{\beta}\cup\leftarrow_{\beta})^{k+1}$时, 要么有$M\onestepbeta P=_\beta N$, 要么有$M\leftarrow_\beta P=_\beta N$, 其中$(P,N)\in(\to_{\beta}\cup\leftarrow_{\beta})^k$. 那么存在$T_0$使得$P\manystepbeta T_0$且$N\manystepbeta T_0$. 由于$\manystepbeta$是传递的, 所以$M\manystepbeta T_0$.
	
	由于$\manystepbeta$的CR性质, $P\manystepbeta M$且$P\manystepbeta T_0$可得到, 存在$T\in\Lambda$使得$M\manystepbeta T$且$T_0\manystepbeta T$. 由于$\manystepbeta$是传递的, 有$N\manystepbeta T_0$且$T_0\manystepbeta T$, 因此$N\manystepbeta T$.
	
	因此对所有的$k\in\mathbb{N}$, 这样的$T$都存在. 命题成立.
\end{proof}
\problem{证明: 若在系统$\lambda\beta$中加入如下公理
$$\textrm{(A)}\qquad\lambda xy.x=\lambda xy.y,$$
则对任何的$M,N\in\Lambda$, $\lambda\beta+(A)\vdash M=N$.}
\begin{proof}
    对于所有的$M,N\in\Lambda$,
	\begin{align*}
		\lambda xy.x=\lambda xy.y&\Rightarrow (\lambda xy.x)MN=(\lambda xy.y)MN\\
		&\Rightarrow M=N
	\end{align*}
\end{proof}
\problem{证明命题3.14.}
命题3.14表述为:

设$R$是$\Lambda$上的一个二元关系, $M\in\mathrm{NF}_R$, 则

(1) 不存在$N\in\Lambda$使得$M\to_RN$;

(2) $M\twoheadrightarrow_R N\Rightarrow M\equiv N$.
\begin{proof}
    (1) 由$\mathrm{NF}_R$的定义可知其成立;
	
	(2) 假设$M\not\equiv N$. $M\twoheadrightarrow_R N$表示存在$M=P_0,\cdots,P_n=N$使得对任意$i<n,P_i\to_R P_{i+1}$. 由于$M\not\equiv N$, $n\geqslant 1$. 但是这样就存在$N$使得$M\to_R N$, 与(1)矛盾. 因此$M\equiv N$.
\end{proof}
\problem{证明引理3.16.}
引理3.16表述为:

若$M\rhd_{\mathrm{mcd}}M'$且$N\rhd_{\mathrm{mcd}}N'$, 则$MN\rhd_{\mathrm{mcd}}M'N'$.
\begin{proof}
    $M\rhd_{\mathrm{mcd}}M'$说明存在序列$M_1,M_2,\cdots,M_n$将$M$归约到$M'$, $N\rhd_{\mathrm{mcd}}N'$同理.
	
	由于$M_i$恒为剩余序列的极小可约式, 所以$MN\rhd_{\mathrm{mcd}}M'N'$.
\end{proof}
\problem{证明定理3.20.}
定理3.20表述如下:

设$M,N\in\Lambda$, 若$M\equivbeta N$, 则存在$T\in\Lambda$使得$M\manystepbeta T$且$N\manystepbeta T$.
\begin{proof}
    同题3.12.
\end{proof}
\problem{试找出$F\in\Lambdaclosed$使$F$ $\lambda$-定义函数$f(x)=3x$.}
\begin{solution}
     构造$F \churchnumber{n} = \churchnumber{3n} = \lambda fx. f^n (f^n (f^n x))$.

\[
 \begin{array}{rcl}
  \lambda fx. f^n (f^n (f^n x)) & = & \lambda fx. f^n (f^n (\churchnumber{n} f x)) \\
  & = & \lambda fx. f^n (\churchnumber{n} f (\churchnumber{n} f x)) \\
  & = & \lambda fx. \churchnumber{n}f (\churchnumber{n} f (\churchnumber{n} f x)) \\
  & = & [\lambda nfx. nf (nf (nfx))] \churchnumber{n}.
 \end{array}
\]

因此, $F \equiv \lambda xyz. xy(xy(xyz))$ $\lambda$-定义函数$f(x)=3x$.
\end{solution}
\problem{令$D\equiv \lambda xyz.z(Ky)x$, 证明: 对于任意的$X,Y\in\Lambda$,
$$\protect\begin{aligned}
    DXY\churchnumber{0}&=X,\\
    DXY\churchnumber{n+1}&=Y.
\protect\end{aligned}
$$
这里$K\equiv \lambda xy.x,\churchnumber{n}\equiv \lambda fx.f^nx$.}
\begin{proof}
     $D XY\churchnumber{0} = \churchnumber{0} (\lambda y. Y) X = (\lambda fx. x) (\lambda y.Y) X = X$, 且
 \[
\begin{array}{rcl}
 DXY\churchnumber{n+1} & = &
  \churchnumber{n+1} (\lambda y. Y) X \\
  &=& (\lambda fx. f^{n+1} x)(\lambda y.Y) X \\
  &=& (\lambda x. (\lambda y.Y)^{n+1} x) X \\ 
  &=& (\lambda x. (\lambda y.Y)^n Y)X \\
  &=& (\lambda x. Y)X = Y.\\
\end{array}
  \]
\end{proof}
\problem{设$\mathrm{Exp}\equiv\lambda xy.yx$, 证明: 对于任意的$n\in\mathbb{N}$和$m\in\mathbb{N}^{*}$,
$$\mathrm{Exp}\churchnumber{n}\churchnumber{m}\equivbeta\churchnumber{n^m}.$$
(Exp由Rosser教授作出)}
\begin{proof}
    将$\mathrm{Exp}$展开, 得:
\[
\begin{array}{rcl}
 \mathrm{Exp} \churchnumber{n} \churchnumber{m} & = &
  (\lambda xy.yx) (\lambda fx.f^n x) (\lambda fx.f^m x) \\
 & = & (\lambda fx.f^m x) (\lambda fx.f^n x) \\
 & = & (\lambda x. (\lambda fy.f^n y)^m x). \\
\end{array}
\]

当$m=1$时, $(\lambda fy.f^ny) x = \lambda y. x^n y$.
假设
$(\lambda fy.f^ny)^m x = \lambda y. x^{n^m} y$, 有
 
\[
 \begin{array}{rcl}
 (\lambda fy.f^ny)^{m+1} x & = &
 (\lambda fy.f^ny) [ (\lambda fy.f^ny)^m x )] \\
 & =& (\lambda fy.f^ny)  (\lambda y.x^{n^m} y) \\
 & = & \lambda y. (\lambda z.x^{n^m} z)^n y \\
 & = & \lambda y. x^{n^{m+1}} y.
  \end{array}
\]

因此, $\mathrm{Exp} \churchnumber{n} \churchnumber{m}
= \lambda xy. x^{n^{m}}y=\churchnumber{n^m}$ 对任意自然数 $m>0$ 都成立.
\end{proof}
\problem{构造$F\in\Lambdaclosed$使得对于任何$n\in\mathbb{N}$,
$$F\churchnumber{n}\equivbeta\churchnumber{2^n}.$$}
\begin{solution}
    $$F\equiv \lambda x.Dx\churchnumber{1}(\mathrm{Exp}\churchnumber{2}x).$$
\end{solution}
\problem{设$f,g:\mathbb{N}\to\mathbb{N}, f=\mathrm{Itw}[g]$, 即
$$\protect\begin{aligned}
   f(0)&=0,\\
   f(n+1)&=g(f(n)),
\protect\end{aligned}$$
且$G\in\Lambdaclosed$ $\lambda$-定义函数$g$. 试求$F\in\Lambdaclosed$使得$F$ $\lambda$-定义函数$f$.}
\begin{proof}
    \[
\begin{array}{rcl}
F\churchnumber{n} & = & D\churchnumber{n}\churchnumber{0}
G[F(\mathrm{pred} \churchnumber{n})] \\
&=  &
\{ \lambda n. Dn\churchnumber{0}
G[F(\mathrm{pred}\quad n)] \} \churchnumber{n} \\
& = &
\{ (\lambda fn.
 \mathrm{D}n\churchnumber{0}
G[f(\mathrm{pred}\quad n)]) F \} \churchnumber{n}. \\
\end{array}
\]

$F\equiv \mathbf{Y}(\lambda xy.
 \mathrm{D}y\churchnumber{0}
G[x(\mathrm{pred}\quad y)])$ $\lambda$-定义函数$f$.
\end{proof}
\problem{证明引理3.39.}

引理3.39表述为:

存在一般递归函数$\mathrm{var,app,abs,num}:\mathbb{N}\to\mathbb{N}$使得:
\begin{enumerate}[label=(\arabic*)]
    \item $\forall n\in\mathbb{N}. \mathrm{var}(n)=\sharp (v^{(n)});$\label{def:var}
    \item $\forall M, N\in \Lambda. \mathrm{app}(\sharp M,\sharp N)=\sharp (MN);$\label{def:app}
    \item $\forall x\in\nabla, M\in\Lambda. \mathrm{abs}(\sharp x, \sharp M) = \sharp(\lambda x. M);$]\label{def:abs}
    \item $\forall n\in\mathbb{N}. \mathrm{num}(n)=\sharp\churchnumber{n}.$\label{def:num}
\end{enumerate}

\begin{proof}
    只要函数$[x,y]$是一般递归函数, 那么由定义3.36可以得到\ref{def:var},\ref{def:app},\ref{def:abs}正确.
    
    对于\ref{def:num}, 
    $$\begin{aligned}
        \mathrm{num}(n)&=\sharp\churchnumber{n}\\
        &=\sharp(\lambda fx.f^nx)\\
        &=\sharp(\lambda f(\lambda x.f^nx))\\
        &=[2,[\sharp f, \sharp (\lambda x.f^nx)]]\\
        &=[2,[\sharp f, [2, [\sharp x, \sharp (f^nx)]]]],
    \end{aligned}$$
    
    $\sharp f, \sharp x$应用\ref{def:var}, $\sharp (f^nx)$还应用\ref{def:app}之后, 可知$\mathrm{num}(n)$也是一般递归函数.
\end{proof}
\problem{证明定理3.41的证明过程中提到的$\lambda$-项$B$的存在性.}
\begin{proof}
We can compute $\mathbf{min}$ by a recursive procedure that keeps a table of variable substitution,
 and log the minimum unused number in each abstraction operation (very complicated, though).
 Since every function in $\PRF$ can be represented in $\lambda$-calculus,
 $\mathbf{min}\in\Lambdaclosed$ exists.
 Also, we have $\mathbf{minus}\in\Lambdaclosed$ such that 
 $\mathbf{minus}\churchnumber{x}\churchnumber{y} = 
 \churchnumber{|x-y|}$. Therefore, let
\[
 B = \lambda fxyz. \mathbf{D}(\mathbf{minus} (\mathbf{min}y) (\mathbf{min}z))x(fz),
\]
$F_{[\churchnumber{x}\mapsto x]} \equivbeta BFx\churchnumber{x}$
is achieved. 



\end{proof}
\problem{构造$H\in\Lambdaclosed$, 使得对于任意$n\in\mathbb{N}, x_1,\cdots, x_n\in\Lambda$, 有$$H\churchnumber{n}x_1\cdots x_n\equivbeta \lambda z.zx_1\cdots x_n.$$}
\begin{proof}
    可知我们需要找到$H$使得$H\churchnumber{n}=\lambda x_1x_2\cdots x_nz. zx_1\cdots x_n$.
    我们可以令$M_n=\lambda x_1x_2\cdots x_nz. zx_1\cdots x_n$, 那么其编码$g(n)=\sharp M_n=[2,[\sharp x_1, \sharp (\lambda x_2\cdots x_n.zx_1\cdots x_n)]]$, 而$\sharp (zx_1\cdots x_n)$是递归的, 因此$g$是递归的. 设$G\in\Lambdaclosed$ $\lambda$-定义了$g$, 从而$G\churchnumber{n}\equivbeta\churchnumber{M_n}$, 因此
    $$E(G\churchnumber{n})\equivbeta E\churchnumber{M_n}\equivbeta M_n$$
    
    取$H\equiv \lambda x.E(Gx)$即可.
\end{proof}
\problem{证明: 存在$H_2\in\Lambdaclosed$, 使得对于任意$F\in\Lambda$, 有$$H_2\churchnumber{F}\equivbeta F\churchnumber{H_2\churchnumber{F}}.$$}
\begin{proof}
    由定理3.41可知, 存在枚举子$E\in\Lambdaclosed$, 使得对于任何$M\in\Lambdaclosed$, 有$E\churchnumber{M}\equivbeta M$.
    
    令$A=\lambda xy.Ey(\churchnumber{xxy})$, $H_2=AA$, 那么有

    \begin{align*}
        H_2\churchnumber{F}&\equiv AA\churchnumber{F}\\
        &\equivbeta E\churchnumber{F}(\churchnumber{AA\churchnumber{F}})\\
        &\equivbeta F(\churchnumber{AA\churchnumber{F}})\\
        &\equiv F\churchnumber{H_2\churchnumber{F}}
    \end{align*}

    因此, 我们找到了使命题成立的$H_2$, 证毕.
\end{proof}
\newpage
\section{组合逻辑}
\problem{求$\lambda^{*}xy.xyy$.}
\begin{solution}

\end{solution}
\problem{令$C\in\mathbb{C}$定义为$$C\equiv S(BBS)(KK).$$ 证明: 对于任意的$X,Y,Z\in\mathbb{C}, CXYZ=XZY$.}
\begin{proof}
\end{proof}
\problem{证明: 若$xP_1\cdots P_m =_{\mathbf{w}} yQ_1\cdots Q_n$, 则$x\equiv y,m\equiv n$, 而且对于任意$i\leq m$, 有$P_i=_{\mathbf{w}}Q_i$.}
\begin{proof}
\end{proof}
\problem{证明: $P=_{\mathbf{w}}Q$当且仅当$\mathrm{CL}_{\mathbf{w}}\vdash P = Q$.}
\begin{proof}
\end{proof}
\problem{证明: $(\lambda^{*}x.M)N=_{\mathbf{w}}M[x:=N].$}
\begin{proof}
\end{proof}
\problem{令$B\equiv S(KS)K$, 证明: 对于任意$P,Q,R\in\mathbb{C}, BPQR=_{\mathbf{w}}P(QR)$.}
\begin{proof}
\end{proof}
\problem{在$\mathrm{CL}$中, 定义$\overline{n}\equiv (SB)^n(KI)$, 其中$n\in\mathbb{N}$, $B$如习题4.6中所定义. 证明: 对于一般递归函数$\varphi:\mathbb{N}\to\mathbb{N}$, 存在组合子$\overline{\varphi}\in\mathbb{C}^{\circ}$, 使得$$\forall n\in\mathbb{N}.\bar{\varphi}\bar{n}=\overline{\varphi(n)}.$$ (此性质由Kleene教授证明)}
\begin{proof}
\end{proof}
\problem{证明: 对于任意$P,Q\in\mathbb{C}$, 若$P=_{\mathbf{w}}Q$, 则$P_{\lambda}\equivbeta Q_{\lambda}$.}
\begin{proof}
\end{proof}
\problem{证明: $(SKK)_{\lambda}\equivbeta \lambda x.x.$}
\begin{proof}
\end{proof}
\problem{证明引理4.13.}

引理4.13表述为:

设$M,N\in\Lambda, x\in V$, 则$$(M[x:=N])_{\mathrm{CL}}\equiv M_{\mathrm{CL}}[x:=N_{\mathrm{CL}}].$$

\begin{proof}
\end{proof}
\newpage
\section{Turing机}
\problem{构造机器计算函数$f(x,y,z)=y$.}
\begin{solution}
$f(x,y,z)=y$由表\ref{tab:sol5.1}定义的机器$\machine{P^3_2}$计算: 
\begin{table}[!htbp]
\centering
\caption{机器$\machine{P^3_2}$}
\label{tab:sol5.1}
\begin{tabularx}{\textwidth}{Y|Y|Y}
\thickhline
    &  0    &      1   \\
\hline
1   & $0R2$ &   $0R1$   \\
\hline
2   & $0R3$ &   $1R2$   \\
\hline
3   & $0L3$ &   $1L4$   \\
\hline
4   & $0L4$ &   $1L5$   \\
\hline
5   & $0R6$ &   $1L5$   \\
\thickhline
\end{tabularx}
\end{table}

对于$\machine{P^3_2}$输入$1:0\underset{\uparrow}{1}^{x+1}01^{y+1}01^{z+1}0\cdots$, 输出$6:0^{x+3}\underset{\uparrow}{1}^{y+1}0\cdots$.
\end{solution}
\problem{构造机器$\machine{copy_1}$使$\machine{copy_1}|0\underset{\uparrow}{1^x}0\cdots \twoheadrightarrow 01^x0\underset{\uparrow}{1^x}0\cdots$.}
\begin{solution}
表\ref{tab:sol5.2}定义了机器$\machine{copy_1}$:
\begin{table}[!htbp]
\centering
\caption{机器$\machine{copy_1}$}
\label{tab:sol5.2}
\begin{tabularx}{\textwidth}{Y|Y|Y}
\thickhline
    &  0    &      1   \\
\hline
1   & $0R6$ &   $0R2$   \\
\hline
2   & $0R3$ &   $1R2$   \\
\hline
3   & $1L4$ &   $1R3$   \\
\hline
4   & $0L5$ &   $1L4$   \\
\hline
5   & $1R1$ &   $1L5$   \\
\thickhline
\end{tabularx}
\end{table}

输入时状态为1,输出时状态为6.
\end{solution}
\problem{构造机器计算函数$f(x,y)=x\times y$.}
\begin{solution}
$f(x,y)=x\times y$由表\ref{tab:sol5.3}定义的机器$\machine{mul}$计算:
\begin{table}[!htbp]
\centering
\caption{机器$\machine{mul}$}
\label{tab:sol5.3}
\begin{tabularx}{\textwidth}{Y|Y|Y}
\thickhline
    &  0    &      1   \\
\hline
1   &    &   $0R2$   \\
\hline
2   & $0R14$ &   $0R3$   \\
\hline
3   & $0R4$ &   $1R3$   \\
\hline
4   & $0R5$ &   $0R5$   \\
\hline
5   & $0R16$ &   $0R6$   \\
\hline
6   & $0R7$ &   $1R6$   \\
\hline
7   & $1L8$ &   $1R7$   \\
\hline
8   & $0L9$ &   $1L8$   \\
\hline
9   & $1R5$   &   $1L9$   \\
\hline
10  & $0L11$ & $1L10$   \\
\hline
11  & $0L12$ &          \\
\hline
12  & $0R14$    & $1L13$    \\
\hline
13  & $0R2$ &   $1L13$  \\
\hline
14  & $0R14$    & $0R15$    \\
\hline
15  & $1O18$    & $0R15$    \\
\hline
16  & $1L17$    & $1L17$    \\
\hline
17  & $0L10$    & $0L10$    \\
\thickhline
\end{tabularx}
\end{table}

\iffalse
由于整个机器比较复杂, 下面解析一下:
状态1消去$x$的首位来到状态2.

状态2在$x$不为0时消去$x$中的1位, 来到状态3; 在$x$为0时, 跳到状态14进行消去$y$的操作.

状态3跳过代表$x$的所有位和间隔$x$和$y$的0后来到状态4.

如果状态4时发现$y$的首位未消去, 消去之, 来到状态5. 此时指向$y$的第一位.

从状态5判断$y$是否已被复制完成, 若是, 来到状态16准备下一次$y$的复制, 若$y$不为0, 则将此位标记为0, 代表准备复制此位, 转状态6. 若$y$为0, 直接视为复制完成.

状态6跳过$y$剩下的位和$y$后面的一个0后来到状态7.

状态7直接滑动到结果最后, 添上1, 复制完成, 进入状态8.

状态8直接向左滑动到结果左边, 并跳过结果前面的0, 进入状态9.

状态9往左滑到准备复制的那位, 补上1, 代表该位复制完成, 进入状态5, 此时机器指向下一位要复制的位, 重复5-9, 直到状态5读到的数为0. 此时$y$的所有位都复制完成.

先讨论状态16. 当状态5读到0后, 才会转到状态16. 这意味着状态5已经遍历了整个$y$, 来到了$y$后面的0. 此时在状态16, 指针指向$y$后面的第2个0, 写1后(首先, 这里如果已经有部分复制操作, 本来的值必为1, 不影响结果. 而如果不写1, 状态14的寻找$y$将在$y$为0时无限进行)左移1位转到状态17, 这一位是$y$后面的0, 左移1位转到状态10. 此时指向$y$的最后一位.

当$y$为0时, 左转2位到状态12, 以此经过$y$被消去的首位1和间隔的0, 到达$x$的最后一位. 当$y$不为0时, 跳过$y$到达$x$的最后一位.

当状态12读到的数为0时, $x$为0. 这意味着$x$已经耗尽, 达到了应有的将$y$复制$x$次的效果, 转状态14准备消去$y$. 否则, 往前滑动, 转状态13.

状态13向前寻找$x$的第一个1. 在发现1前面的0后, 右移1位以指向$x$的第一位1, 转状态2进行消去, 消去一次, 代表新一轮复制的开始.

状态14负责找到$y$. 在状态12和状态2因为$x$的值为0而转到状态14后, 状态14略过之后的0, 从而找到第一个1, 这个第一个1可能是$y$消去首位后的第一位, 可能是$y$为0, 导致状态5转到状态16, 使状态16写了1. 找到第一个1后, 转状态15.

状态15负责抹去$y$. 对于$y$为0的情况, 状态15将这个1抹去后在其后面一位写1, 表示结果为0. 对于$y$不为0的情况, 后面的序列是表示$y$的序列和表示结果的序列, 两者中间有1个0间隔开来, 但此时表示结果的序列仅仅是$x\times y$个1, 还需要再添加1个1. 状态15将$y$的连续1序列抹去后, 将本用来间隔的0写成1, 并指向这个位置, 便得到了结果.
\fi
对于$\machine{mul}$输入$1:0\underset{\uparrow}{1}^{x+1}01^{y+1}0\cdots$, 在$y=0$时输出$18:0^{x+6}\underset{\uparrow}{1}0\cdots$, 在$y\neq 0$时输出$18:0^{x+y+4}\underset{\uparrow}{1}^{x\times y + 1}0\cdots$.
\end{solution}
\problem{构造机器计算函数$f(x)=2^x$.}
\begin{solution}
由定理5.13的证明过程, 可如此构造:

令$f(x,y)=2^xy$, 令$y$恒为1, $g(x)=2x$, 即可计算$f(x)=2^x$.

首先, 构造出初始值($y$)1, 使用机器$M_1$, 定义如表\ref{tab:sol5.4m1}:
\begin{table}[!htbp]
\centering
\caption{题5.4机器$M_1$}
\label{tab:sol5.4m1}
\begin{tabularx}{\textwidth}{Y|Y|Y}
\thickhline
    &  0    &      1   \\
\hline
1   & $0R2$ &   $1R1$   \\
\hline
2   & $1R3$ &           \\
\hline
3   & $1L4$ &           \\
\hline
4   &       &   $1L5$   \\
\hline
5   & $0L6$ &           \\
\hline
6   & $0R7$ &   $1L6$   \\
\thickhline
\end{tabularx}
\end{table}

易知$M_1|1:0\underset{\uparrow}{1}^{x+1}00\cdots\twoheadrightarrow7:0\underset{\uparrow}{1}^{x+1}01100\cdots$.

定义机器$M_2$为表\ref{tab:sol5.4m2}:
\begin{table}[!htbp]
\centering
\caption{题5.4机器$M_2$}
\label{tab:sol5.4m2}
\begin{tabularx}{\textwidth}{Y|Y|Y}
\thickhline
    &  0    &      1   \\
\hline
1   &       &   $0R2$   \\
\hline
2   & $0Ru$ &   $1R3$   \\
\hline
3   & $0R4$ &   $1R3$   \\
\thickhline
\end{tabularx}
\end{table}

易知$x>0$时$M_2|1:0\underset{\uparrow}{1}^{x+1}01100\cdots\twoheadrightarrow 4:001^{x}0\underset{\uparrow}{1}100\cdots$ (在$x=0$时输出为$u:000\underset{\uparrow}{1}100\cdots$).

令$M_3=M_2\concat \machine{double} + 3\concat \machine{compress}\concat \machine{shiftl}$, $M_4=\mathrm{repeat}M_3$, 机器$\machine{f}=M_1\concat M_4$为所求.

而整个的机器如表\ref{tab:sol5.4}所示.
\begin{table}[!htbp]
\centering
\caption{题5.4机器$\machine{f}$}
\label{tab:sol5.4}
\begin{tabularx}{\textwidth}{Y|Y|Y}
\thickhline
    &       0   &       1   \\
\hline
1   &   $0R2$   &   $1R1$   \\
\hline
2   &   $1R3$   &           \\
\hline
3   &   $1L4$   &           \\
\hline
4   &           &   $1L5$   \\
\hline
5   &   $0L6$   &           \\
\hline
6   &   $0R7$   &   $1L6$   \\
\hline
7   &           &   $0R8$   \\
\hline
8   &   $0R27$  &   $1R9$   \\
\hline
9   &   $0R10$  &   $1R9$   \\
\hline
10  &           &   $1R11$  \\
\hline
11  &   $0R12$  &   $1R11$  \\
\hline
12  &   $1R13$  &   $1R12$  \\
\hline
13  &   $1R14$  &           \\
\hline
14  &   $0L15$  &           \\
\hline
15  &   $0L16$  &   $1L15$  \\
\hline
16  &   $0R17$  &   $1L16$  \\
\hline
17  &   $0R18$  &   $1O10$  \\
\hline
18  &           &   $0R19$  \\
\hline
19  &           &   $1L20$  \\
\hline
20  &   $1L21$  &   $0R25$  \\
\hline
21  &   $0R22$  &   $1R20$  \\
\hline
22  &   $0L23$  &   $1R22$  \\
\hline
23  &           &   $0L24$  \\
\hline
24  &   $0R19$  &   $1L24$  \\
\hline
25  &   $0L26$  &   $1L25$  \\
\hline
26  &   $0R7$   &   $1L26$  \\
\thickhline
\end{tabularx}
\end{table}

\end{solution}
\problem{设机器$M_1$定义如表5.24.\\
对于输入$\bar{x}$, 求输出.}
\begin{table}[H]
\centering
\caption*{\textbf{表5.24}}
\begin{tabularx}{\textwidth}{Y|Y|Y}
\thickhline
    &  0    &      1   \\
\hline
1   & $0L3$ &   $1R2$   \\
\hline
2   & $0L3$ &   $0R1$   \\
\hline
3   & $0L3$ &   $1L3$   \\
\thickhline
\end{tabularx}
\end{table}

\begin{solution}
$M_1$的操作为如果读到1, 则交替写入1和0并令被指位置向右一位, 换句话说, 把第偶数次读到的1写成0, 读到0后一直向左倒带到最左边无法读取从而停机. 因此当其输入为$\bar{x}$时, 输出为$\overline{(\underbrace{0,0,\cdots,0}_{\left\lceil \frac{x+1}{2}\right\rceil\text{个}0})}$.
\end{solution}
\problem{设机器$M_2$定义如表5.25.\\
对于输入$(2,1):01^n01^m01^k00\cdots$, 其中$n,m,k\in\mathbb{N}^+$, 求输出.}
\begin{table}[H]
\centering
\caption*{\textbf{表5.25}}
\begin{tabularx}{\textwidth}{Y|Y|Y}
\thickhline
&0&1\\
\hline
1&$0R2$&$0R1$\\
\hline
2&$1R3$&$0R1$\\
\hline
3&$1R4$&\\
\hline
4&$1R5$&\\
\hline
5&$1L6$&\\
\hline
6&$0R7$&$1L6$\\
\thickhline
\end{tabularx}
\end{table}

\begin{solution}
对于输入$(2,1):01^n01^m01^k00\cdots$, 机器$M_2$有如下计算过程:
\begin{align}
    1:&0\underset{\uparrow}{1}^n01^m01^k00\cdots\tag{0R1}\\
    1:&0^{n+1}\underset{\uparrow}{0}1^m01^k00\cdots\tag{0R2}\\
    2:&0^{n+2}\underset{\uparrow}{1}^{m}01^k00\cdots\tag{0R1}
\end{align}
当$m=1$时, 计算过程为:
\begin{align}
    1:&0^{n+3}\underset{\uparrow}{0}1^k00\cdots\tag{0R2}\\
    2:&0^{n+4}\underset{\uparrow}{1}^k00\cdots\tag{0R1}
\end{align}
当$m>1$时, 计算过程为:
\begin{align}
    1:&0^{n+3}\underset{\uparrow}{1}^{m-1}01^k00\cdots\tag{0R1}\\
    1:&0^{n+m+2}\underset{\uparrow}{0}1^k00\cdots\tag{0R2}\\
    2:&0^{n+m+3}\underset{\uparrow}{1}^k00\cdots\tag{0R1}
\end{align}

此时机器都处于状态为2, 输入为$0^{n+m+3}\underset{\uparrow}{1}^k00\cdots$的时刻.

当$k=1$时, 计算过程与$m=1$时类似, 当$k>1$时, 计算过程与$m>1$时类似, 机器都会处于状态2, 输入为$0^{n+m+k+4}\underset{\uparrow}{0}0\cdots$的时刻. 接下来的计算过程为:
\begin{align}
    2:&0^{n+m+k+4}\underset{\uparrow}{0}0\cdots\tag{1R3}\\
    3:&0^{n+m+k+4}1\underset{\uparrow}{0}0\cdots\tag{1R4}\\
    4:&0^{n+m+k+4}11\underset{\uparrow}{0}0\cdots\tag{1R5}\\
    5:&0^{n+m+k+4}111\underset{\uparrow}{0}0\cdots\tag{1L6}\\
    6:&0^{n+m+k+4}11\underset{\uparrow}{1}10\cdots\tag{1L6}\\
    6:&0^{n+m+k+3}\underset{\uparrow}{0}11110\cdots\tag{0R7}\\
    7:&0^{n+m+k+3}\underset{\uparrow}{1}1110\cdots\tag{停}
\end{align}

因此机器的输出就是$7:0^{n+m+k+3}\underset{\uparrow}{1}1110\cdots$, 即计算函数$f(x,y,z)=3$.
\end{solution}
\problem{构造机器计算函数$f(x)=\lfloor\sqrt{x}\rfloor$.}
\begin{solution}
令$y=\lfloor\sqrt{x}\rfloor$, 那么有$y\leqslant \sqrt{x}<y+1$, 有$y^2\leqslant x<(y+1)^2$.

因此$f(x)=\mu y.[x < (y+1)^2]$, 即表示在给定$x$的情况下使得$x<(y+1)^2$恒成立的最小$y$值. 由于$x,(y+1)^2\in\mathbb{N}$, 所以$x+1\leqslant (y+1)^2$.

所以我们有$f(x)=\mu y. [S(x)\dotdiv \mathrm{sq}(S(y))]$. 其中, $\mathrm{sq}(x)=x^2$.

函数sq可以通过$\machine{copy_1}\concat \machine{shiftl} \concat \machine{mul}$计算, 机器$\machine{mul}$即表\ref{tab:sol5.3}定义的机器, 计算$S$函数的机器$\machine{S}$已在书中给出, 计算$x\dotdiv y$的机器$\machine{sub}$见题5.11.

因此总体思路如下:
输入为$\overline{x}$, 整个转化过程为:

$$\overline{x}\to \overline{(x+1, y)}\to \overline{(x+1, y, x+1, y)} \to \overline{(x+1, y, f(x+1,y))}$$

当$f(x,y)$为0时, 抹掉$f(x,y)$和$x$, 指向$y$. 否则, 抹掉$f(x,y)$, $y$加1, 再复制一次$x,y$.

先定义机器$M_0$如表\ref{tab:sol5.7m0}:
\begin{table}[!htbp]
\centering
\caption{题5.7机器$M_0$}
\label{tab:sol5.7m0}
\begin{tabularx}{\textwidth}{Y|Y|Y}
\thickhline
    &       0   &       1   \\\hline
1   &   $1R2$   &   $1R1$   \\\hline
2   &   $0R3$   &           \\\hline
3   &   $1L4$   &           \\\hline
4   &   $0L4$   &   $1L5$   \\\hline
5   &   $0R6$   &   $1L5$   \\
\thickhline
\end{tabularx}
\end{table}

易知$M_0|1:0\underset{\uparrow}{1}^{x+1}0\cdots\twoheadrightarrow 6:0\underset{\uparrow}{1}^{x+2}010\cdots$, 这完成了$\overline{x}\to \overline{(x+1, y)}$的步骤.

$\overline{(x+1, y)}\to \overline{(x+1, y, x+1, y)}\to \overline{(x+1, y, f(x+1,y))}$的步骤由$\machine{copy_2}^2\concat \machine{f}\concat\machine{compress}$完成. $\machine{f}$计算函数$f(x,y)=x\dotdiv \mathrm{sq}(S(y))$(注意不是$S(x)$而是$x$, 因为加1操作已经在$M_0$完成), $\machine{f}=\machine{shiftr}\concat\machine{S}\concat\machine{copy_1}\concat\machine{shiftl}\concat\machine{mul}\concat\machine{compress}\concat\machine{shiftl}\concat\machine{sub}$.

再定义机器$M_1$如表\ref{tab:sol5.7m1}:
\begin{table}[!htbp]
\centering
\caption{题5.7机器$M_1$}
\label{tab:sol5.7m1}
\begin{tabularx}{\textwidth}{Y|Y|Y}
\thickhline
    &       0   &       1   \\\hline
1   &           &   $0R2$   \\\hline
2   &   $0L8$   &   $0R3$   \\\hline
3   &   $0L4$   &   $0R3$   \\\hline
4   &   $0L4$   &   $1R5$   \\\hline
5   &   $1L6$   &           \\\hline
6   &   $0L7$   &   $1L6$   \\\hline
7   &   $0R12$  &   $1L7$   \\\hline
8   &   $0L8$   &   $1L9$   \\\hline
9   &   $0L10$  &   $1L9$   \\\hline
10  &   $0R11$  &   $0L10$  \\\hline
11  &   $0R11$  &   $1Ou$   \\
\thickhline
\end{tabularx}
\end{table}

易知:

$M_1|1:01^{x+1}01^{y+1}0\underset{\uparrow}{1}0\cdots\twoheadrightarrow u:0^{x+3}\underset{\uparrow}{1}^{y+1}00\cdots$

$M_1|1:01^{x+1}01^{y+1}0\underset{\uparrow}{1}^{z+1}0\cdots\twoheadrightarrow 12:0\underset{\uparrow}{1}^{x+1}01^{y+2}00\cdots(z>0)$

那么, 令$M_2=\machine{copy_2}^2\concat\machine{f}\concat\machine{compress}\concat M_1$, $M_3=\mathrm{repeat}M_2$, 则机器$M=M_0\concat M_3$可以计算$f(x)=\lfloor \sqrt{x}\rfloor$.
\end{solution}
\problem{设机器$\machine{f_1}$计算函数$f_1$, 机器$\machine{f_2}$计算函数$f_2$, 这里$f_1,f_2$为一元数论函数. 构造机器$\machine{f}$计算函数$f(x)=f_1(x)+f_2(x)$.}
\begin{solution}
$\machine{f}=\machine{copy_1}\concat \machine{f_1} \concat \machine{compress} \concat \machine{shiftl} \concat \machine{copy_2} \concat \machine{shiftr} \concat \machine{f_2} \concat\machine{compress} \concat\machine{shiftl}^2 \concat \machine{erase} \concat\machine{add}.$
\end{solution}
\problem{设$f(x)=h(g_1(x),g_2(x),g_3(x))$, 试由机器$\machine{g_1},\machine{g_2},\machine{g_3}$和$\machine{h}$构造机器$\machine{f}$.}
\begin{solution}
\begin{align*}
    \machine{f}&=\machine{copy_1}\concat \machine{g_1} \concat \machine{compress} \concat \machine{shiftl}\\
    &\concat \machine{copy_2} \concat \machine{shiftr} \concat \machine{g_2} \concat \machine{compress} \concat \machine{shiftl}^2 \\
    &\concat \machine{copy_3} \concat \machine{shiftr}^2 \concat \machine{g_3} \concat \machine{compress} \concat \machine{shiftl}^3 \\
    &\concat \machine{erase} \concat \machine{h}.
\end{align*}

\end{solution}
\problem{设$f:\mathbb{N}\to\mathbb{N}$定义如下:
$$\protect\begin{aligned}
f(0)&=0,\\
f(x+1)&=g(f(x)).
\protect\end{aligned}$$
证明: 若$g$为Turing-可计算, 则$f$为Turing-可计算.}
\begin{proof}
按照定理5.14的证明方式, 令$y$恒为0. 因此首先在原来的输入上添加0作为$y$, 构建机器$M_1$如表\ref{tab:sol5.10m1}:

\begin{table}[!htbp]
\centering
\caption{题5.10机器$M_1$}
\label{tab:sol5.10m1}
\begin{tabularx}{\textwidth}{Y|Y|Y}
\thickhline
    &  0    &      1   \\
\hline
1   & $0R2$ &   $1R1$   \\
\hline
2   & $1L3$ &           \\
\hline
3   & $0L4$ &           \\
\hline
4   & $0R5$ &   $1L4$   \\
\thickhline
\end{tabularx}
\end{table}

易知$M_1|1:0\underset{\uparrow}{1}^{x+1}0\cdots\twoheadrightarrow5:0\underset{\uparrow}{1}^{x+1}010\cdots$.

由于$g$是Turing-可计算的, 所以存在机器$\machine{g}$计算函数$g$.

构造机器$M_2$如表\ref{tab:sol5.10m2}所示:
\begin{table}[!htbp]
\centering
\caption{题5.4机器$M_2$}
\label{tab:sol5.10m2}
\begin{tabularx}{\textwidth}{Y|Y|Y}
\thickhline
    &  0    &      1   \\
\hline
1   &       &   $0R2$   \\
\hline
2   & $0Ru$ &   $1R3$   \\
\hline
3   & $0R4$ &   $1R3$   \\
\thickhline
\end{tabularx}
\end{table}

令$M_3=M_2\concat \machine{g}+3 \concat \machine{compress} \concat \machine{shiftl}$, 并令$\machine{f}=\mathrm{repeat}M_3$, $M=M_1\concat \machine{f}$即为能计算$f$的机器, 因此$f$为Turing-可计算.
\end{proof}
\problem{构造机器计算函数$f(x,y)=x\dotdiv y$.}
\begin{solution}
基本思想: $x$和$y$每回各消去1, 直到有一个为0为止.
$f(x,y)=x\dotdiv y$由表\ref{tab:sol5.11}定义的机器$\machine{sub}$计算:
\begin{table}[!htbp]
\centering
\caption{题5.11机器$\machine{sub}$}
\label{tab:sol5.11}
\begin{tabularx}{\textwidth}{Y|Y|Y}
\thickhline
    &       0   &       1   \\\hline
1   &           &   $0R2$   \\\hline
2   &   $0R8$   &   $1R3$   \\\hline
3   &   $0R4$   &   $1R3$   \\\hline
4   &   $0R4$   &   $0R5$   \\\hline
5   &   $0L10$  &   $1L6$   \\\hline
6   &   $0L6$   &   $1L7$   \\\hline
7   &   $0R1$   &   $1L7$   \\\hline
8   &   $0R8$   &   $0R9$   \\\hline
9   &   $1O13$  &   $0R9$   \\\hline
10  &   $0L10$  &   $1R11$  \\\hline
11  &   $1L12$  &           \\\hline
12  &   $0R13$  &   $1L12$  \\\hline
\thickhline
\end{tabularx}
\end{table}

对于$\machine{sub}$输入$1:0\underset{\uparrow}{1}^{x+1}01^{y+1}0\cdots$, 在$x\leqslant y$时输出$13:0^{x+y+4}\underset{\uparrow}{1}0\cdots$, 在$x>y$时输出$13:0^{y+2}\underset{\uparrow}{1}^{x-y+1}0\cdots$.
\end{solution}
\problem{证明: $\mathrm{Even}=\{2x:x\in\mathbb{N}\}$是Turing-可计算的.}
\begin{proof}
只要能够构造出这样的机器$M$, 它满足
\begin{align*}
    M|1:0\underset{\uparrow}{1}^{2x}0\cdots&\twoheadrightarrow u:0\cdots0\underset{\uparrow}{1}10\cdots\\
    M|1:0\underset{\uparrow}{1}^{2x+1}0\cdots&\twoheadrightarrow v:0\cdots0\underset{\uparrow}{1}0\cdots
\end{align*}

构造如表\ref{tab:sol5.12}所示:
\begin{table}[!htbp]
\centering
\caption{题5.12机器$M$}
\label{tab:sol5.12}
\begin{tabularx}{\textwidth}{Y|Y|Y}
\thickhline
    &       0   &       1   \\\hline
1   &   $1R3$   &   $0R2$   \\\hline
2   &   $1O4$   &   $0R1$   \\\hline
3   &   $1L4$   &           \\
\thickhline
\end{tabularx}
\end{table}
\end{proof}
\problem{证明: $S=\{a_1,a_2,\cdots,a_k\}$是Turing-可计算的.}
\begin{proof}
先将$S$中的元素按照升序排列, 即$a_i<a_j\Leftrightarrow i<j$.

构造的机器将拥有$a_k+4$个状态. 状态1遇到0停机(不可能), 遇到1写0转状态2. 在状态的值$s\leqslant a_{k}+2$时, 遇到0, 则写1, 若$s-2\in S$, 不动并停机, 否则右移一位转状态$a_k+4$; 遇到1, 则抹去, 右移一位, 转状态$s+1$.

$s=a_{k}+3$时, 当遇到0时, 写1, 右移1位, 转状态$a_{k}+4$; 当遇到1时, 写0, 右移1位, 不改变状态.

$s=a_{k}+4$时, 当遇到0时, 写1, 左移1位, 停机; 当遇到1时, 直接停机.

示意如表\ref{tab:sol5.13}.

\begin{table}[!htbp]
\centering
\caption{题5.13机器示例}
\label{tab:sol5.13}
\begin{tabularx}{\textwidth}{Y|Y|Y}
\thickhline
    &       0   &       1   \\\hline
1   &           &   $0R2$   \\\hline
... &   ...     &   ...     \\\hline
$a_i+2$   &   $1O(a_k+5)$   &     $0R(a_i+3)$      \\\hline
... &   ...     &   ...     \\\hline
$m$   &   $1R(a_k+4)$   &   $0R(m+1)$ \\\hline
... &   ...     &   ...     \\\hline
$a_k+2$   &   $1O(a_k+5)$ &   $0R(a_k+3)$ \\\hline
$a_k+3$   &   $1R(a_k+4)$ &   $0R(a_k+3)$ \\\hline
$a_k+4$   &   $1L(a_k+5)$ &   \\\hline
\thickhline
\end{tabularx}
\end{table}
\end{proof}
\problem{设$f:\mathbb{N}\to\mathbb{N}$是Turing-可计算的, 构造机器$M$使其输出$f$的最小零点.}
\begin{proof}
    基本思路为: 从0开始计算f的值, 一旦发现f的值为0, 输出, 否则加1再算.
    
    这里不知道$M$的输入是什么, 假定$M$的输入为$0\underset{\uparrow}{1}000\cdots$, 即自然数0. 令计算$f$的机器为$\machine{f}$.
    
    定义机器$M_1$如表\ref{tab:sol5.14m1}:
\begin{table}[!htbp]
\centering
\caption{题5.14机器$M_1$}
\label{tab:sol5.14m1}
\begin{tabularx}{\textwidth}{Y|Y|Y}
\thickhline
    &       0   &       1   \\\hline
1   &           &   $0R2$   \\\hline
2   &   $0L3$   &   $0R5$   \\\hline
3   &   $0L3$   &   $1L4$   \\\hline
4   &   $0Ru$   &   $1L4$   \\\hline
5   &   $0L6$   &   $0R5$   \\\hline
6   &   $0L6$   &   $1R7$   \\\hline
7   &   $1L8$   &           \\\hline
8   &   $0R9$   &   $1L8$   \\
\thickhline
\end{tabularx}
\end{table}

易知:

$M_1|1:01^{x+1}0\cdots 0\underset{\uparrow}{1}0\cdots\twoheadrightarrow u:0\underset{\uparrow}{1}^{x+1}0\cdots$

当$y>0$时, $M_1|1:01^{x+1}0\cdots 0\underset{\uparrow}{1}^{y+1}0\cdots\twoheadrightarrow 9:0\underset{\uparrow}{1}^{x+2}0\cdots$

到达$u$状态表示无论如何都停机. (可以改成$0R7$, 因为下一位肯定是1)

那么令$M_2=\machine{copy_1}\concat \machine{f}\concat M_1$, 则$M=\mathrm{repeat}M_2$为所求. 在函数$f$没有零点时, 机器$M$将永不停机.
\end{proof}
\problem{证明定理5.21中函数$g$为一般递归函数.}
\begin{proof}
	这里需要使用题5.17的结论.
	
	当$m\notin S$时, 有$g(m)=0$.
	
	否则, 可以解码出编码$m$对应的机器$M$的行数和各个行的内容.
	
	而$M_1$计算常值函数$m$的机器可通过$\machine{Z}\concat \machine{S}^m$构造, 显然可以通过$m$由一般递归函数计算.
	
	而$\hat{M}=M_1\concat M$, 这意味着$M$的所有行都要自加$m$, 这个映射也是一般递归的.
	
	因此最终整个函数都是一般递归的.
\end{proof}
\problem{证明引理5.25中的函数$e(m,l)$为初等函数.}
\begin{proof}
	利用题5.17的结论, 由$m$可以通过一般递归函数计算每一行的内容$\sharp j$, $\sharp x$, $\sharp y$, $\sharp z$, $\sharp u$, $\sharp v$, $\sharp w$.
	
	同时, 由$l$也可以通过一般递归函数计算纸带的编码长度$t$和纸带位置$(j,k):a_1,a_2,\cdot,a_t$.
	
	通过循环, 比较, 分支等步骤, 可以计算出$\sharp d(a_j, k),\sharp p(a_j, k),\sharp s(a_j, k)$.
	
	因此$e(m,l)=\langle\sharp d(a_j,k),\sharp p(a_j, k),\sharp s(a_j, k)\rangle\in\GRF$.
\end{proof}
\problem{令$S=\{\sharp M:M\text{为Turing机}\}$, 证明$S$为Turing-可计算.}
\begin{proof}
	这相当于用一个Turing机来解码一个数判断其是否为合法的Turing机编码.
	
	首先确定机器的行数$k$, $k=\max a\leqslant \sharp M.P(a-1)|\sharp M$. $P(n)$代表第$n$个素数, $n$从0开始计数, 即$P(0)=2, P(1)=3, P(2)=5, \cdots$.
	
	然后是每一行的编码, $r_i=\ep(i,\sharp M)$.
	
	然后是对行内的元素进行解码, 对于机器的一行$\begin{array}{|c|c|c|}\hline j&xyz&uvw\\\hline\end{array}$, 可得
	
	$$\sharp j = \ep(0,r_i), \sharp x=\ep(1,r_i), \cdots, \sharp w=\ep(6,r_i)$$
	
	每个数都会得到一个结果, 但是结果需要合法:
	
	首先, 1和2肯定不合法.
	
	如果解析得到某行$\sharp x=2$, 则应该有$\sharp y=\sharp z=2$, 即$LLL$, 否则, $\sharp x < 2\land \sharp y\in\{2,3,4\}$. 其余情况均不合法.
	
	如果解析得到某行$\sharp u=4$, 则应该由$\sharp v=\sharp w=4$, 即$RRR$, 否则, $\sharp u < 2\land \sharp v\in\{2,3,4\}$. 其余情况均不合法.
	
	$\forall m,n\leqslant k, m\neq n. \sharp j(r_m)\neq \sharp_j(r_n)$. 即每一行的标号都不同.
	
	这里已经明确给出了计算流程, 显然我们可以通过编程在有限时间内输出某个自然数是否合法的Turing机编码, 其自然是Turing-可计算的.
\end{proof}
\problem{由CT证明函数$g(n)$可计算, 这里$$g(n)=\text{在自然对数之底}\mathrm{e}\text{的十进制展开式中第}n\text{个数字}.$$}
\begin{proof}
根据泰勒公式, 

$$\mathrm{e}^x=\sum_{i=0}^n\frac{1}{i!}+\frac{\mathrm{e}^\theta}{(n+1)!},\quad 0<\theta<1.$$

令$f(n)=\lfloor \mathrm{e}\cdot n!\rfloor$, 我们证明$f\in\PRF$.

$f(0)=f(1)=2$, 当$n\geqslant 2$时, $f(n)=\sum_{i=0}^n\frac{n!}{i!}+\frac{\mathrm{e}^\theta}{n+1}$, 而此时$0<\frac{\mathrm{e}^\theta}{n+1}<1$, 所以$f(n)=\sum_{i=0}^n\frac{n!}{i!}$. 显然$f(n)\in\PRF$.

令$h(n)=\lfloor \mathrm{e}\cdot n\rfloor$, 那么有$h(0)=0$, $h(n)=\left\lfloor\frac{f(n)}{(n-1)!}\right\rfloor(n>0)$, 因此$h(n)\in\PRF$.

那么$g(1)=2$, $g(n)=h(10^n)\dotdiv h(10^{n\dotdiv 1})\cdot 10(n>0)$, 因此$g(n)\in\PRF$, 所以$g(n)$可计算.
\end{proof}
\problem{(1) 什么是停机问题?\\
(2) 什么是可判定问题 (decision problem)?\\
(3) 停机问题可判定吗?}
\begin{solution}
    \begin{enumerate}[label=(\arabic*)]
        \item 是否存在能行过程来判定机器对所有输入皆停机?
        \item 设$A$为$\mathbb{N}$的子集, $A$是可判定的指$A$的特征函数$\chi_A$是Turing-可计算的, 即有机器$M_A$, 其对于输入$\bar{x}$, 若$x\in A$, 则输出$\bar{0}$; 否则, 输出$\bar{1}$.
        \item 停机问题不可判定.
    \end{enumerate}
\end{solution}
\problem{(1) 什么是通用Turing机 (universal Turing machine)?\\
(2) 通用Turing机起什么作用?}
\begin{solution}
    \begin{enumerate}[label=(\arabic*)]
        \item 机器$U$是通用Turing机, 当其满足对任何机器$M$和任何$(n_1,\cdots,n_k)\in\mathbb{N}^k$,
        $$M|\overline{(n_1,\cdots,n_k)}\twoheadrightarrow \overline{y}\Leftrightarrow U|\overline{(\sharp M, n_1, \cdots, n_k)}\twoheadrightarrow \overline{y}.$$
        \item 其凭自身就能完成任何Turing机可能做到的任何事, 可以模拟任何其他Turing机, 在早期程序储存式计算机的研制中起到了重要的促进作用.
    \end{enumerate}
\end{solution}
\newpage
\end{document}