\problem{构造机器计算函数$f(x,y)=x\times y$.}
\begin{solution}
$f(x,y)=x\times y$由表\ref{tab:sol5.3}定义的机器$\machine{mul}$计算:
\begin{table}[!htbp]
\centering
\caption{机器$\machine{mul}$}
\label{tab:sol5.3}
\begin{tabularx}{\textwidth}{Y|Y|Y}
\thickhline
    &  0    &      1   \\
\hline
1   &    &   $0R2$   \\
\hline
2   & $0R14$ &   $0R3$   \\
\hline
3   & $0R4$ &   $1R3$   \\
\hline
4   & $0R5$ &   $0R5$   \\
\hline
5   & $0R16$ &   $0R6$   \\
\hline
6   & $0R7$ &   $1R6$   \\
\hline
7   & $1L8$ &   $1R7$   \\
\hline
8   & $0L9$ &   $1L8$   \\
\hline
9   & $1R5$   &   $1L9$   \\
\hline
10  & $0L11$ & $1L10$   \\
\hline
11  & $0L12$ &          \\
\hline
12  & $0R14$    & $1L13$    \\
\hline
13  & $0R2$ &   $1L13$  \\
\hline
14  & $0R14$    & $0R15$    \\
\hline
15  & $1O18$    & $0R15$    \\
\hline
16  & $1L17$    & $1L17$    \\
\hline
17  & $0L10$    & $0L10$    \\
\thickhline
\end{tabularx}
\end{table}

\iffalse
由于整个机器比较复杂, 下面解析一下:
状态1消去$x$的首位来到状态2.

状态2在$x$不为0时消去$x$中的1位, 来到状态3; 在$x$为0时, 跳到状态14进行消去$y$的操作.

状态3跳过代表$x$的所有位和间隔$x$和$y$的0后来到状态4.

如果状态4时发现$y$的首位未消去, 消去之, 来到状态5. 此时指向$y$的第一位.

从状态5判断$y$是否已被复制完成, 若是, 来到状态16准备下一次$y$的复制, 若$y$不为0, 则将此位标记为0, 代表准备复制此位, 转状态6. 若$y$为0, 直接视为复制完成.

状态6跳过$y$剩下的位和$y$后面的一个0后来到状态7.

状态7直接滑动到结果最后, 添上1, 复制完成, 进入状态8.

状态8直接向左滑动到结果左边, 并跳过结果前面的0, 进入状态9.

状态9往左滑到准备复制的那位, 补上1, 代表该位复制完成, 进入状态5, 此时机器指向下一位要复制的位, 重复5-9, 直到状态5读到的数为0. 此时$y$的所有位都复制完成.

先讨论状态16. 当状态5读到0后, 才会转到状态16. 这意味着状态5已经遍历了整个$y$, 来到了$y$后面的0. 此时在状态16, 指针指向$y$后面的第2个0, 写1后(首先, 这里如果已经有部分复制操作, 本来的值必为1, 不影响结果. 而如果不写1, 状态14的寻找$y$将在$y$为0时无限进行)左移1位转到状态17, 这一位是$y$后面的0, 左移1位转到状态10. 此时指向$y$的最后一位.

当$y$为0时, 左转2位到状态12, 以此经过$y$被消去的首位1和间隔的0, 到达$x$的最后一位. 当$y$不为0时, 跳过$y$到达$x$的最后一位.

当状态12读到的数为0时, $x$为0. 这意味着$x$已经耗尽, 达到了应有的将$y$复制$x$次的效果, 转状态14准备消去$y$. 否则, 往前滑动, 转状态13.

状态13向前寻找$x$的第一个1. 在发现1前面的0后, 右移1位以指向$x$的第一位1, 转状态2进行消去, 消去一次, 代表新一轮复制的开始.

状态14负责找到$y$. 在状态12和状态2因为$x$的值为0而转到状态14后, 状态14略过之后的0, 从而找到第一个1, 这个第一个1可能是$y$消去首位后的第一位, 可能是$y$为0, 导致状态5转到状态16, 使状态16写了1. 找到第一个1后, 转状态15.

状态15负责抹去$y$. 对于$y$为0的情况, 状态15将这个1抹去后在其后面一位写1, 表示结果为0. 对于$y$不为0的情况, 后面的序列是表示$y$的序列和表示结果的序列, 两者中间有1个0间隔开来, 但此时表示结果的序列仅仅是$x\times y$个1, 还需要再添加1个1. 状态15将$y$的连续1序列抹去后, 将本用来间隔的0写成1, 并指向这个位置, 便得到了结果.
\fi
对于$\machine{mul}$输入$1:0\underset{\uparrow}{1}^{x+1}01^{y+1}0\cdots$, 在$y=0$时输出$18:0^{x+6}\underset{\uparrow}{1}0\cdots$, 在$y\neq 0$时输出$18:0^{x+y+4}\underset{\uparrow}{1}^{x\times y + 1}0\cdots$.
\end{solution}